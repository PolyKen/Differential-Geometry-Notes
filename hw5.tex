\documentclass{article}

\usepackage{changepage,aligned-overset,amsfonts,amssymb,amsthm,enumerate,geometry}
\usepackage{xeCJK}
\geometry{a4paper, scale = 0.8}
\setCJKmainfont{STSong}
\begin{document}
%\author{谢铮 15338200}
\title{Homework 5}
\date{Mar 22, 2019}
\maketitle

\setlength\parindent{0em}   % cancel all indent
\setlength\parskip{1.0\baselineskip} % set skip between paragraphs

\par
\textbf{4-2 Ex.2}\\
Prove the following "converse" of Prop.1: Let $\phi: S \to \bar S$ be an isometry
and $\mathcal{X}: U \to S$ a parametrization at $p \in S$; then
$\bar{\mathcal{X}} = \phi \circ \mathcal{X}$ is a parametrization at $\phi(p)$ and
$\bar E = E$, $\bar F = F$, $\bar G = G$.

\par
\textbf{\textit{Solution.}}\\
Since $\phi$ is an isometry from $S$ to $\bar S$, it is also a diffeomorphism thus
$\bar{\mathcal{X}} = \phi \circ \mathcal{X}$ is a parametrization at $\phi(p)$.\\
Moreover, for any $w_1, w_2 \in T_p(S)$, 
$$
    \langle w_1, w_2 \rangle = \langle d\phi_p(w_1), d\phi_p(w_2) \rangle
$$
Particularly,
$$
    \langle \mathcal{X}_u, \mathcal{X}_v \rangle 
    = \langle d\phi_p(\mathcal{X}_u), d\phi_p(\mathcal{X}_v) \rangle
    = \langle \bar{\mathcal{X}}_u, \bar{\mathcal{X}}_u \rangle
$$
Thus $E = \bar E$. Similarly we can obtain $F = \bar F$, $G = \bar G$. \qed

\par
\textbf{4-2 Ex.3}\\
Show that a diffeomorphism $\phi: S \to \bar S$ is an isometry if and only if the
arc length of any parametrized curve in $S$ is equal to the arc length of the
image curve by $\phi$.

\par
\textbf{\textit{Solution.}}\\
$"\Rightarrow":$
Suppose that $\phi: S \to \bar S$ is an isometry, then the first fundamental form
keeps invariant under $\phi$ and thus the arc length, which can be calculated by
first fundamental form, is also invariant. \qed

\par
$"\Leftarrow":$
Pick any parametrized curve $\alpha: I \to S, \alpha(t) = \mathcal{X}(u(t), v(t))$, 
let $\beta = \phi \circ \alpha$,
suppose that $s_\alpha = s_\beta$, i.e.
$$
    \int_0^T |\alpha'(t)|dt = \int_0^T |\beta'(t)|dt
$$
since $\alpha$ is arbitrary, it follows that
$$
    |\alpha'(0)| = |\beta'(0)|
$$
which can be written as first fundamental form
$$
    \sqrt{Eu'^2 + 2Fu'v' + Gv'^2} = \sqrt{\bar Eu'^2 + 2\bar Fu'v' + \bar Gv'^2}
$$
for all $u', v'$. Therefore $E = \bar E$, $F = \bar F$, $G = \bar G$. \qed

\par
\textbf{4-2 Ex.5}\\
Let $\alpha_1: I \to \mathbb{R}^3$, $\alpha_2: I \to \mathbb{R}^3$
be regular parametrized curves, where the parameter is the 
arc length. Assume that the curvatures $k_1$ of $\alpha_1$
and $k_2$ of $\alpha_2$ satisfy $k_1(s) = k_2(s) \neq 0, s\in I$,
let
$$
    \mathcal{X}_1(s,v) = \alpha_1(s) + v\alpha'_1(s)
$$
$$
    \mathcal{X}_2(s,v) = \alpha_2(s) + v\alpha'_2(s)
$$
be their regular tangent surfaces and let $V$ be a neighborhood
of $(s_0, v_0)$ such that $\mathcal{X}_1(V) \subset \mathbb{R}^3$,
$\mathcal{X}_2(V) \subset \mathbb{R}^3$ are regular surfaces.
Prove that 
$\mathcal{X}_1 \circ \mathcal{X}_2^{-1}: \mathcal{X}_2(V) \to \mathcal{X}_1(V)$
is an isometry.

\par
\textbf{\textit{Solution.}}\\
Let $k(s) = k_1(s) = k_2(s)$
Note that
$$
    {\mathcal{X}_1}_s = \alpha_1'(s) + vk_1(s)n_1(s), {\mathcal{X}_1}_v = \alpha_1'(s)
$$
$$
    {\mathcal{X}_2}_s = \alpha_2'(s) + vk_2(s)n_2(s), {\mathcal{X}_2}_v = \alpha_2'(s)
$$
Thus,
$$
\begin{aligned}
    E_1 &= \langle \alpha_1'(s) + vk_1(s)n_1(s), \alpha_1'(s) + vk_1(s)n_1(s) \rangle
    = |\alpha_1'(s)|^2 + v^2k_1^2(s) 
    = 1 + v^2k^2(s)\\
    F_1 &= \langle \alpha_1'(s) + vk_1(s)n_1(s), \alpha_1'(s) \rangle
    = |\alpha_1'(s)|^2 = 1\\
    G_1 &= \langle \alpha_1'(s), \alpha_1'(s) \rangle
    = |\alpha_1'(s)|^2 = 1
\end{aligned}
$$
$$
\begin{aligned}
    E_2 &= \langle \alpha_2'(s) + vk_2(s)n_2(s), \alpha_2'(s) + vk_2(s)n_2(s) \rangle
    = |\alpha_2'(s)|^2 + v^2k_2^2(s) 
    = 1 + v^2k^2(s)\\
    F_2 &= \langle \alpha_2'(s) + vk_2(s)n_2(s), \alpha_2'(s) \rangle
    = |\alpha_2'(s)|^2 = 1\\
    G_2 &= \langle \alpha_2'(s), \alpha_2'(s) \rangle
    = |\alpha_2'(s)|^2 = 1
\end{aligned}
$$
which shows that $\phi = \mathcal{X}_1 \circ \mathcal{X}_2^{-1}$ is an isometry. \qed

\par
\textbf{4-2 Ex.6}\\
Let $\alpha: I \to \mathbb{R}^3$ be a regular parametrized curve with $k(t) \neq 0$,
$t \in I$. Let $\mathcal{X}(t,v)$ be its tangent surface. Prove that, for each
$(t_0, v_0) \in I \times (R - \{0\})$, there exists a neighborhood $V$ of $(t_0, v_0)$
such that $\mathcal{X}(V)$ is isometric to an open set of the plane(thus, tangent
surfaces are locally isometric to planes).

\par
\textbf{\textit{Solution.}}\\
Suppose $\alpha$ is parametrized by arc length.
Then consider a plane curve $\beta$, which is also parametrized by arc length.
And let $k_\beta(s) = k_\alpha(s)$, then by the conclusion of $Ex.5$, the tangent
surface of $\alpha$ is isometric to the tangent surface of $\beta$, which is an
open set of a plane since $\beta$ is a plane curve. \qed

\par
\textbf{4-2 Ex.8}\\
Let $G: \mathbb{R}^3 \to \mathbb{R}^3$ be a map such that
$$
    |G(p) - G(q)| = |p-q|
$$
for all $p,q \in \mathbb{R}^3$.\\
(that is, $G$ is a distance-preserving map). Prove that there exists 
$p_0 \in \mathbb{R}^3$ and a linear isometry $F$ of the vector space $\mathbb{R}^3$
such that
$$
    G(p) = F(p) + p_0
$$
for all $p \in \mathbb{R}^3$.

\par
\textbf{\textit{Solution.}}\\
Let $p_0 = G(0)$ and define $F(p) = G(p) - p_0$. Then
$$
    |F(p) - F(q)| = |G(p) - G(q)| = |p - q|
$$
Particularly, let $q = 0$, we get $|F(p)| = |p|$. And therefore $F(0)=0$.
Note that for each $w_1, w_2 \in \mathbb{R}^3$,
$$
\begin{aligned}
    |F(w_1 + w_2)|^2 - |F(w_1 - w_2)|^2 &= \langle w_1 + w_2, w_1 + w_2 \rangle - \langle w_1 - w_2, w_1 - w_2 \rangle\\
    &= 4\langle w_1, w_2 \rangle
\end{aligned}
$$

\end{document}