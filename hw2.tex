\documentclass{article}

\usepackage{changepage,aligned-overset,amsfonts,amssymb,amsthm,enumerate,geometry}
\usepackage{xeCJK}
\geometry{a4paper, scale = 0.8}
\setCJKmainfont{STSong}
\begin{document}
\author{谢铮 15338200}
\title{Homework 2}
\date{Mar 22, 2019}
\maketitle

\setlength\parindent{0em}   % cancel all indent
\setlength\parskip{1.0\baselineskip} % set skip between paragraphs

\par
\textbf{2-2 Ex.1}\\
Show that the cylinder $\{(x,y,z) \in \mathbb{R}^3: x^2+y^2=1\}$ is a regular surface, and find parametrizations whose
coordinate neighborhoods cover it.

\par
\textbf{\textit{Solution.}}\\
For each point $p = (x_0, y_0, z_0) \in S = \{(x,y,z) \in \mathbb{R}^3: x^2+y^2=1\}$, without loss of generality we can 
assume that $y_0 > 0$, then let $q = (x_0, 0, z_0)$ and $U$ a sufficiently small neighborhood of $q$ in $xOz$ plane.
Define $\mathcal{X}: U \to S$ by $\mathcal{X}(x,z)=(x,\sqrt{1-x^2}, z)$, it's clearly that
$\mathcal{X}$ is differentiable, one-to-one, and its inverse is also continuous. Moreover,
$$
    d\mathcal{X} = 
    \left(
        \begin{array}{cc}
            1 & 0\\\\
            \dfrac{x}{\sqrt{1-x^2}} & 0\\\\
            0 & 1
        \end{array}
    \right)
$$
has rank 2 at any point $(x_0, z_0)$. Thus $\mathcal{X}$ is a parametrization in $U$. \qedsymbol

\par
\textbf{2-2 Ex.9}\\
Let $V$ be an open set in the $xy$ plane. Show that the set
$$
    \{(x,y,z) \in \mathbb{R}^3: z=0,(x,y) \in V\}
$$
is a regular surface.\\

\par
\textbf{\textit{Solution.}}\\
Let $S$ denote the set above and define $\mathcal{X}: V \to S$ by
$$
    \mathcal{X}(x,y) = (x,y,0)
$$
Then it's easy to verify that $\mathcal{X}$ satisfies condition 1, 2 and 3. So $\mathcal{X}$ is
a parametrization and $S$ is a regular surface. \qedsymbol

\par
\textbf{2-2 Ex.12}\\
Show that $\mathcal{X}: U \subset \mathbb{R}^2 \to \mathbb{R}^3$ given by 
$$
    \mathcal{X}(u,v) = (a \sin u \cos v, b \sin u \sin v, c \cos u), a, b, c \neq 0,
$$
where $0<u<\pi$, $0<v<2 \pi$ is a parametrization for the ellipsoid
$$
    \frac{x^2}{a^2} + \frac{y^2}{b^2} + \frac{z^2}{c^2} = 1
$$
Describe geometrically the curves $u = const.$ on the ellipsoid.\\

\par
\textbf{\textit{Solution.}}\\
It's easy to see that $\mathcal{X}$ is a diffeomorphism, also note that
$$
    d\mathcal{X} = 
    \left(
        \begin{array}{cc}
            a \cos u \cos v & -a \sin u \sin v\\\\
            b \cos u \sin v & b \sin u \cos v\\\\
            -c \sin u & 0
        \end{array}
    \right)
$$
always has rank 2 for all $u \in (0, \pi), v \in (0, 2 \pi)$. Therefore $\mathcal{X}$ is
a parametrization.\\
The curves $u = const.$ is a set of ellipses with axes $2a \sin u$ and $2b \sin u$. \qedsymbol

\par
\textbf{2-2 Ex.16}\\
One way to define a system of coordinates for the sphere $S^2$, given by $x^2 + y^2 + (z-1)^2 = 1$,
is to consider the so-called stereographic projection $\pi: S^2 / \{N\} \to \mathbb{R}^2$ which
carries a point $p = (x,y,z)$ of the sphere $S^2$ minus the north pole $N = (0,0,2)$
onto the intersection of the $xy$ plane with the straight line which connects $N$ to $p$.
Let $(u,v) = \pi(x,y,z)$, where $(x,y,z) \in S^2 / \{N\}$ and $(u,v) \in xy$ plane.\\
\textbf{a. }Show that $\pi^{-1}: \mathbb{R}^2 \to S^2$ is given by 
$$
    \pi^{-1} = 
    \begin{cases}
        x = \frac{4u}{u^2+v^2+4}\cr
        y = \frac{4v}{u^2+v^2+4}\cr
        z = \frac{2u^2+2v^2}{u^2+v^2+4}
    \end{cases}
$$
\textbf{b. }Show that it is possible, using stereographic projection, to cover the sphere with
two coordinate neighborhoods.

\par
\textbf{\textit{Solution.}}\\
\textbf{a. }By definition of $\pi$, $(0,0,2), (x,y,z)$ and $(u,v,0)$ are in the same line. Hence we have
$$
    (x,y,z-2) = \lambda (u,v,-2), \lambda \in \mathbb{R}
$$
Also we know that $x^2 + y^2 + (z-1)^2 = 1$, it follows that
$$
    \lambda^2 u^2 + \lambda^2 v^2 + (2\lambda - 1)^2 = 1
$$
$$
    \lambda^2 u^2 + \lambda^2 v^2 + 4\lambda^2 - 4\lambda = 0
$$
Note that $\lambda \neq 0$ because $(x,y,z) \neq N$, so we have
$$
    \lambda = \frac{4}{u^2+v^2+4}
$$
Therefore,
$$
    x = \lambda u = \frac{4u}{u^2+v^2+4}, y = \lambda v = \frac{4v}{u^2+v^2+4}, z = 2-2\lambda = \frac{2u^2+2v^2}{u^2+v^2+4} \qed
$$

\textbf{b. }The conclusion above shows that for each point $p$ in $S^2 / \{N\}$, we can find
a corresponding parametrization. Particularly, for $p=N$, consider map $\mathcal{X}: B(0,1) \to S^2$ defined by
$$
    \mathcal{X}(x,y) = (x,y,1+\sqrt{1-x^2-y^2}), (x,y) \in B(0,1)
$$
where $B(0,1)$ is the unit ball in $\mathbb{R}^2$. It's easy to verify that $\mathcal{X}$ is a parametrization. Hence we can 
cover $S^2$ using two coordinate neighborhoods. \qedsymbol

\end{document}