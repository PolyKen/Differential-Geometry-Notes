\documentclass{article}

\usepackage{changepage,aligned-overset,amsfonts,amssymb,amsthm,enumerate,geometry}

\geometry{a4paper, scale = 0.8}

\author{Zheng Xie}
\title{Curves}
\date{May 28, 2018}

\begin{document}
\maketitle

\setlength\parindent{0em}   % cancel all indent
\setlength\parskip{1.0\baselineskip} % set skip between paragraphs

\par
\textbf{Definition 1.1 (parametrized differentiable curve)}\\
A parametrized differentiable curve is a map $\alpha :I \to \mathbb R^3$, 
where $I = (a,b)$ is an open interval, $\alpha(t) = (x(t),y(t),z(t))$ and $x(t),y(t),z(t)$ are differentiable.

\par
A parametrized differentiable curve is not necessarily one-to-one.

\par
\textbf{Definition 1.2 (regular parametrized differentiable curve)}\\
A parametrized differentiable curve is called \textbf{regular} if $\alpha'(t) \neq 0$ for all $t \in I$.
If $|\alpha'(t)| \equiv 1$, then we say that $\alpha(t)$ is parametrized by arc length.

\par
From now on we shall consider only the regular parametrized differentiable curves.
Usually we write regular curves in short.

\par
\textbf{Remark.}\\
If $\alpha(t)$ is a regular curve, it can always be reparametrized by arc length.
\par
Given $t \in I$, the arc length of a regular curve $\alpha:I \to \mathbb R^3$ from the point $t_0$ is
$$
    s(t) = \int_{t_0}^t |{\alpha'(t)}|dt
$$
where
$$
    |{\alpha'(t)}| = \sqrt{x'(t)^2+y'(t)^2+z'(t)^2} > 0
$$
So the inverse function $t = t(s)$ exists and it follows
$$
    |\frac{d}{ds}\alpha(t(s))| = |\alpha'(t) \cdot t'(s)| = |\alpha'(t)| \cdot \frac{dt}{ds} = 1
$$

\par
To simplify our exposition, we shall restrict ourselves to curves parametrized by arc length, 
actually this restriction is not essential.

\par
\textbf{Definition 1.3 (orientation)}\\
Two ordered bases $e = \{e_i\}$ and $f = \{f_i\}$, $i=1,2,...,n$, of an n-dimensional vector space $V$ have the
same orientation if the matrix of change of basis have positive determinant. We denote this relation by $e \sim f$.
It's easy to see that $\sim$ is an equivalence relation.\\
If $det(e_1, e_2, e_3) > 0$, then base $e = \{e_i\}$ is positively oriented. Similarly, if $det(e_1, e_2, e_3) < 0$,
then base $e = \{e_i\}$ is negatively oriented.

\par
The set of all ordered basis of $V$ is thus decomposed into two equivalence classes.

\par
\textbf{Definition 1.4 (vector product)}\\
Let $u, v \in \mathbb{R}^3$, the vector product of $u$ and $v$ is the unique vector $u \wedge v \in \mathbb{R}^3$
characterized by
$$
    (u \wedge v) \cdot w = det(u, v, w), \quad \text{for all }w \in \mathbb{R}^3
$$
Or, equivalently,\\
if $u$ and $v$ are linearly dependent,
$$
    u \wedge v = 0
$$
otherwise, $u \wedge v$ is the unique vector such that
$$
    (u \wedge v) \cdot u = 0,\quad(u \wedge v) \cdot v = 0
$$
$$
    |u \wedge v| = |u||v|\sin \langle u, v \rangle
$$
$$
    det(u, v, u \wedge v) > 0
$$

\par
According to the definition of vector product,
$$
    (u \wedge v) \cdot w = det(u, v, w) = 
    \left|\begin{array}{ccc} 
            u_1 & v_1 & w_1 \\ 
            u_2 & v_2 & w_2\\ 
            u_3 & v_3 & w_3
    \end{array}\right|
$$
Take $w = e_i$ (natural basis) respectively, we have
$$
    u \wedge v = \Biggl(
    \left|\begin{array}{cc} 
        u_2 & v_2 \\ 
        u_3 & v_3
    \end{array}\right|,
    \left|\begin{array}{cc} 
        u_3 & v_3 \\ 
        u_1 & v_1
    \end{array}\right|,
    \left|\begin{array}{cc} 
        u_1 & v_1 \\ 
        u_2 & v_2
    \end{array}\right|
    \Biggr)
$$

\par
\textbf{Property 1.1 (vector product)}\\
1. (anticommutativity) $u \wedge v = - v \wedge u$.\\
2. $(au + bv) \wedge w = au \wedge w + bv \wedge w$.\\
3. $u \wedge v = 0$ if and only if $u$ and $v$ are linearly dependent.\\
4. $(u \wedge v) \cdot u = 0$, $(u \wedge v) \cdot v = 0$.\\

\par
\textbf{Property 1.2 (vector product)}\\
Given arbitrary vectors $u, v, x, y \in \mathbb{R}^3$,
$(u \wedge v) \cdot (x \wedge y) = 
\left|\begin{array}{cc} 
    u \cdot x & v \cdot x \\ 
    u \cdot y & v \cdot y
\end{array}\right|$.\\
\textbf{\textit{Proof.}}\\
$$
\begin{aligned}
    (u \wedge v) \cdot (x \wedge y) &= 
    \Biggl(
    \left|\begin{array}{cc} 
        u_2 & v_2 \\ 
        u_3 & v_3
    \end{array}\right|,
    \left|\begin{array}{cc} 
        u_3 & v_3 \\ 
        u_1 & v_1
    \end{array}\right|,
    \left|\begin{array}{cc} 
        u_1 & v_1 \\ 
        u_2 & v_2
    \end{array}\right|
    \Biggr)
    \cdot
    \Biggl(
    \left|\begin{array}{cc} 
        x_2 & y_2 \\ 
        x_3 & y_3
    \end{array}\right|,
    \left|\begin{array}{cc} 
        x_3 & y_3 \\ 
        x_1 & y_1
    \end{array}\right|,
    \left|\begin{array}{cc} 
        x_1 & y_1 \\ 
        x_2 & y_2
    \end{array}\right|
    \Biggr)\\
    &=
    (u_2v_3-u_3v_2)(x_2y_3-x_3y_2) +
    (u_3v_1-u_1v_3)(x_3y_1-x_1y_3) +
    (u_1v_2-u_2v_1)(x_1y_2-x_2y_1)\\
    &=
    (u_2v_3x_2y_3 + u_3v_2x_3y_2 + u_3v_1x_3y_1 + u_1v_3x_1y_3 + u_1v_2x_1y_2 + u_2v_1x_2y_1)\\
    &-(u_3v_2x_2y_3 + u_2v_3x_3y_2 + u_1v_3x_3y_1 + u_3v_1x_1y_3 + u_2v_1x_1y_2 + u_1v_2x_2y_1)\\
    &= 
    (u_1x_1 + u_2x_2 + u_3x_3)(v_1y_1 + v_2y_2 + v_3y_3)-(u_1y_1 + u_2y_2 + u_3y_3)(v_1x_1 + v_2x_2 + v_3x_3)\\
    &=
    \left|\begin{array}{cc} 
        u \cdot x & v \cdot x \\ 
        u \cdot y & v \cdot y
    \end{array}\right| \quad \qedsymbol\\
\end{aligned}
$$

\par
It follows immediately,
$$
    |u \wedge v|^2 = (u \wedge v)\cdot(u \wedge v) = |u|^2|v|^2(1 - \cos^2 \langle u,v\rangle) = |A|^2
$$

\par 
\textbf{Property 1.2 (vector product)}\\
Given arbitrary vectors $u,v,w \in \mathbb{R}^3$, $(u \wedge v) \wedge w = (u \cdot w) \wedge v - (v \cdot w) \wedge u$.\\
\textbf{\textit{Proof.}}\\
First we observe that both sides are linear in $u,v,w$, thus the identity will be true if it holds for
all basis vectors $e_1, e_2, e_3$. This last verification is straightforward, for example,
$$
    (e_1 \wedge e_2) \wedge e_3 = e_3 \wedge e_3 = 0 = (e_1 \cdot e_3) \wedge e_2 - (e_2 \cdot e_3) \wedge e_1
$$

\par
\textbf{Property 1.3 (vector product)}\\
Let $u, v$ be parametrized differentiable curves in $\mathbb{R}^3$, then
$$
    \frac{d}{dt}(u(t) \wedge v(t)) = u'(t) \wedge v(t) + u(t) \wedge v'(t)
$$
\textbf{\textit{Proof.}}\\
Suppose $u(t) = (x_1(t), y_1(t), z_1(t))$, $v(t) = (x_2(t), y_2(t), z_2(t))$.
Then
$$
\begin{aligned}
    u(t) \wedge v(t) 
    &= 
    \Biggl(
    \left|\begin{array}{cc} 
        u_2(t) & v_2(t) \\ 
        u_3(t) & v_3(t)
    \end{array}\right|,
    \left|\begin{array}{cc} 
        u_3(t) & v_3(t) \\ 
        u_1(t) & v_1(t)
    \end{array}\right|,
    \left|\begin{array}{cc} 
        u_1(t) & v_1(t) \\ 
        u_2(t) & v_2(t)
    \end{array}\right|
    \Biggr)\\
    &=
    (u_2(t)v_3(t)-u_3(t)v_2(t), u_3(t)v_1(t)-u_1(t)v_3(t), u_1(t)v_2(t)-u_2(t)v_1(t))\\
    &=
    (u_2(t)v_3(t), u_3(t)v_1(t), u_1(t)v_2(t)) - (u_3(t)v_2(t), u_1(t)v_3(t), u_2(t)v_1(t))
\end{aligned}
$$
The following verification is easy. \quad \qedsymbol

\par
\textbf{Definition 1.3 (curvature)}\\
Let $\alpha: I \to \mathbb R^3$ be a RPDC parametrized by arc length $s \in I$. $k(s) = |\alpha''(s)|$ is
called the \textbf{curvature} of $\alpha$ at $s$.

\par
If $\alpha$ is a straight line, $\alpha(s) = us + v$, then $k \equiv 0$. It's also true conversely.

\par
\textbf{Proposition 1.1}\\
Let $\alpha: I \to \mathbb R^3$ be a RPDC parametrized by arc length $s \in I$. For any $s \in I$,
there exists a unit vector $n(s)$ such that $n(s)$ is normal to tangent vector $\alpha'(s)$ and
$\alpha''(s) = k(s)n(s)$. $n(s)$ is called the \textbf{normal vector} at $s$. The plane determined by the
(unit) tangent vector $\alpha'(s)$ and normal vector $n(s)$ is called the \textbf{osculating plane} at $s$.

\par
\textbf{\textit{Proof.}}\\
It suffices to show that $\alpha''(s)$ is normal to $\alpha'(s)$.\\
Since $\alpha'(s) = 1$, $\alpha'(s) \cdot \alpha'(s) = 1$\\
By differentiating it we obtain $\alpha'(s) \cdot \alpha''(s) = 0$.

\par
At points where $k(s) = 0$, the normal vector (and therefore the osculating plane) is not defined. It is
therefore convenient to say that $s \in I$ is a \textbf{singular point of order 1} if $\alpha''(s) = 0$ 
(in this context, the points where $\alpha'(s) = 0$ are called singular points of order 0).\\
In what follows, we shall restrict ourselves to RPDC parametrized by arc length without singular points
of order 1. We shall denote by $t(s) = \alpha'(s)$ the (unit) tangent vector of $\alpha$ at $s$. Thus,
$t'(s) = k(s)n(s)$.

\par
\textbf{Definition 1.4 (binormal vector)}\\
Let $\alpha: I \to \mathbb R^3$ be a RPDC parametrized by arc length $s \in I$. Given $s$, 
$b(s) = t(s) \land n(s)$ is called the \textbf{binormal vector} at $s$, where $t(s)$ is the tangent vector, 
$n(s)$ is the normal vector. 

\par
By definition, binormal vector $b(s)$ is normal to the osculating plane. Since $b(s)$ is a unit vector,
the length $|b'(s)|$ measures the rate of change of the neighboring osculating planes with the osculating
plane at $s$. That is, $b'(s)$ measures how rapidly the curve pulls away from the osculating plane at $s$,
in a neighborhood of $s$.

\par
Then we want to compute $b'(s)$. Since $|b(s)| = 1$, $b'(s)$ is normal to $b(s)$ \textbf{(1)}. \\
On the other hand,
$$
\begin{aligned}
    b'(s) &= t'(s) \land n(s) + t(s) \land n'(s) \\ &= k(s)n(s) \land n(s) + t(s) \land n'(s) \\ &= t(s) \land n'(s)
\end{aligned}
$$
That is, $b'(s)$ is normal to $t(s)$ \textbf{(2)}. By \textbf{(1)(2)} we can conclude that $b'(s)$ is parallel to 
$n(s)$.\\
Thus we can write 
$$
    b'(s) = \tau (s)n(s)
$$

\par
\textbf{Definition 1.5 (torsion)}\\
Let $\alpha: I \to \mathbb R^3$ be a RPDC parametrized by arc length such that $\alpha''(s) \neq 0$, $s \in I$.\\
The number $\tau(s)$ defined by $b'(s) = \tau (s)n(s)$ is called the \textbf{torsion} of $\alpha$ at $s$.

\par
If $\alpha$ is a plane curve, i.e. $\alpha(I)$ is in a plane, then the plane of the curve agrees with the 
osculating plane. Hence $\tau \equiv 0$. Conversely, if $\tau \equiv 0$ (and $k(s) \neq 0$), then $b(s) \equiv b_0$
is a constant, and therefore
$$
(\alpha (s) \cdot b_0)' = \alpha '(s) \cdot b_0 = 0
$$
It follows that $\alpha(s) \cdot b_0 = c$ where $c$ is a constant. Hence $\alpha(s)$ is contained in a plane normal
to $b_0$. The condition that $k \neq 0$ is essential here.

\par
In contrast to the curvature, the torsion may be either positive or negative. Notice that by changing orientation
the binormal vector changes sign, since $b(s) = t(s) \land n(s)$. It follows that $b'(s)$, and, therefore, the
torsion, remain invariant under a change of orientation.

\par
To compute $n'(s)$, note that $n(s) = b(s) \land t(s)$, we have
$$
    n'(s) = b'(s) \land t(s) + b(s) \land t'(s) = -\tau (s)b(s) -k(s)t(s)
$$

\par
\textbf{Theorem 1.1 (fundamental theorem of the local theory of curves)}\\
Given differentiable functions $k(s)>0$ and $\tau (s)$, $s \in I$, there exists a regular parametrized curve
$\alpha:I \to \mathbb R^3$ such that $s$ is the arc length, $k(s)$ is the curvature, and $\tau (s)$ is the torsion
of $\alpha$. Moreover, any other curve $\bar{\alpha}$, satisfying the same conditions, differs from $\alpha$ by a
rigid motion, that is, there exists an orthogonal linear map $\rho$ of $\mathbb R^3$ with positive determinant
and a vector $c$ such that $\bar{\alpha} = \rho \circ \alpha + c$.

\par
\textbf{Remark}\\
Given a regular parametrized curve $\alpha:I \to \mathbb R^3$ (not necessarily parametrized by arc length), it is
possible to obtain a curve $\beta:J \to \mathbb R^3$ parametrized by arc length which has the same trace as 
$\alpha$. In fact, Let
$$
    s = s(t) = \int_{t_0}^{t}|\alpha'(t)|dt,\quad t,t_0 \in I
$$
Since $\frac{ds}{dt} = |\alpha'(t)| \neq 0$, the function $s = s(t)$ has a differentiable inverse
$t = t(s)$, $s \in s(I) = J$, where, by an abuse of notation, $t$ also denotes the inverse function $s^{-1}$ of
$s$. Now set $\beta = \alpha \circ t:J \to \mathbb R^3$. It's clear that $\beta(J) = \alpha(I)$ and
$|\beta'(s)| = |\alpha'(t) \cdot \frac{dt}{ds}| = 1$. This shows that $\beta$ has the same trace as $\alpha$ and
is parametrized by arc length. It is usual to say that $\beta$ is a reparametrization of $\alpha(I)$ by arc
length.\\
This fact allows us to extend all local concepts previously defined to regular curves with an arbitrary parameter.
Thus, we say that the curvature $k(t)$ of $\alpha:I \to \mathbb R^3$ at $t \in I$ is the curvature of a
reparametrization $\beta: J \to \mathbb R^3$ of $\alpha(I)$ by arc length at the corresponding point $s = s(t)$.
This is clearly independent of the choice of $\beta$ and shows that the restriction of considering only curves
parametrized by arc length is not essential.

\par
\textbf{Definition 1.6 (closed plane curve, simple curve)}\\
A \textbf{closed plane curve} is a RPDC $\alpha:[a,b] \to \mathbb R^2$ such that $\alpha$ and all its derivatives
agree at $a$ and $b$. That is,
$$
    \alpha(a) = \alpha(b), \alpha'(a) = \alpha'(b), \alpha''(a) = \alpha''(b), \cdots
$$
The curve $\alpha$ is \textbf{simple} if it has no further self-intersections. That is, if $t_1, t_2 \in [a,b)$, 
$t_1 \neq t_2$, then $\alpha(t_1) \neq \alpha(t_2)$.

\par
We assume that a simple closed curve $C$ in the plane bounds a region of this plane that is called the 
\textbf{interior} of $C$. This is part of the Jordan curve theorem. Whenever we speak of the area bounded by a
simple closed curve $C$, we mean the area of the interior of $C$. We assume further that the parameter of a
simple closed curve can be so chosen that if one is going along the curve in the direction of increasing paramters,
then the interior of the curve remains to the left. Such a curve will be called \textbf{positively oriented}.

\par
\textbf{Theorem 1.2 (the isoperimetric inequality)}\\
Let $C$ be a simple closed plane curve with length $l$, and let $A$ be the area of the region bounded by $C$.
Then
$$
    l^2 - 4\pi A \geq 0
$$
and equality holds if and only if $C$ is a circle.

\par
\textbf{Definition 1.7 (vertex)}\\
A \textbf{vertex} of a regular plane curve $\alpha:[a,b] \to \mathbb R^2$ is a point $t \in [a,b]$ where
$k'(t) = 0$.

\par
For instance, an ellipse with unequal axes has exactly four vertices, namely the points where the axes meet the
ellipse.

\par
\textbf{Theorem 1.3 (the four-vertex theorem)}\\
A simple closed convex curve has at least four vertices.

\par
A straight line $L$ in the plane is determined by the distance $\rho \geq 0$ from $L$ to the origin $O$ of the
coordinates and by the angle $\theta$, $0 \leq \theta < 2\pi$ which a half-line starting at $O$ and normal to
$L$ makes with the $x$ axis. 

\par
\textbf{Theorem 1.4 (the Cauchy-Crofton formula)}\\
Let $C$ be a regular plane curve with length $l$. The measure of the set of straight lines (counted with
multiplicities) which meet $C$ is equal to $2l$. That is,
$$
    \iint n d\rho d\theta = 2l
$$
where $n = n(\rho,\theta)$ is the number of intersection points of the straight line $(\rho,\theta)$ with $C$.

\end{document}