\documentclass{article}

\usepackage{changepage,aligned-overset,amsfonts,amssymb,amsthm,enumerate,geometry}

\geometry{a4paper, scale = 0.8}

\author{Zheng Xie}
\title{Curves}
\date{May 28, 2018}

\begin{document}
\maketitle

\setlength\parindent{0em}   % cancel all indent
\setlength\parskip{1.0\baselineskip} % set skip between paragraphs

\par
\textbf{Definition 1.1 (parametrized differentiable curve)}\\
A parametrized differentiable curve is a map $\alpha :I \to \mathbb R^3$, 
where $I = (a,b)$ is an open interval, $\alpha(t) = (x(t),y(t),z(t))$ and $x(t),y(t),z(t)$ are differentiable.

\par
A parametrized differentiable curve is not necessarily one-to-one.

\par
\textbf{Definition 1.2 (regular parametrized differentiable curve)}\\
A parametrized differentiable curve is called \textbf{regular} if $\alpha'(t) \neq 0$ for all $t \in I$.
If $|\alpha'(t)| \equiv 1$, then we say that $\alpha(t)$ is parametrized by arc length.

\par
From now on we shall consider only the regular parametrized differentiable curves.
Usually we write regular curves in short.

\par
\textbf{Remark.}\\
If $\alpha(t)$ is a regular curve, it can always be reparametrized by arc length.
\par
Given $t \in I$, the arc length of a regular curve $\alpha:I \to \mathbb R^3$ from the point $t_0$ is
$$
    s(t) = \int_{t_0}^t |{\alpha'(t)}|dt
$$
where
$$
    |{\alpha'(t)}| = \sqrt{x'(t)^2+y'(t)^2+z'(t)^2} > 0
$$
So the inverse function $t = t(s)$ exists and it follows
$$
    |\frac{d}{ds}\alpha(t(s))| = |\alpha'(t) \cdot t'(s)| = |\alpha'(t)| \cdot \frac{dt}{ds} = 1
$$

\par
To simplify our exposition, we shall restrict ourselves to curves parametrized by arc length, 
actually this restriction is not essential.

\par
\textbf{Definition 1.3 (orientation)}\\
Two ordered bases $e = \{e_i\}$ and $f = \{f_i\}$, $i=1,2,...,n$, of an n-dimensional vector space $V$ have the
same orientation if the matrix of change of basis have positive determinant. We denote this relation by $e \sim f$.
It's easy to see that $\sim$ is an equivalence relation.\\
If $det(e_1, e_2, e_3) > 0$, then base $e = \{e_i\}$ is positively oriented. Similarly, if $det(e_1, e_2, e_3) < 0$,
then base $e = \{e_i\}$ is negatively oriented.

\par
The set of all ordered basis of $V$ is thus decomposed into two equivalence classes.

\par
\textbf{Definition 1.4 (vector product)}\\
Let $u, v \in \mathbb{R}^3$, the vector product of $u$ and $v$ is the unique vector $u \wedge v \in \mathbb{R}^3$
characterized by
$$
    (u \wedge v) \cdot w = det(u, v, w), \quad \text{for all }w \in \mathbb{R}^3
$$
Or, equivalently,\\
if $u$ and $v$ are linearly dependent,
$$
    u \wedge v = 0
$$
otherwise, $u \wedge v$ is the unique vector such that
$$
    (u \wedge v) \cdot u = 0,\quad(u \wedge v) \cdot v = 0
$$
$$
    |u \wedge v| = |u||v|\sin \langle u, v \rangle
$$
$$
    det(u, v, u \wedge v) > 0
$$

\par
According to the definition of vector product,
$$
    (u \wedge v) \cdot w = det(u, v, w) = 
    \left|\begin{array}{ccc} 
            u_1 & v_1 & w_1 \\ 
            u_2 & v_2 & w_2\\ 
            u_3 & v_3 & w_3
    \end{array}\right|
$$
Take $w = e_i$ (natural basis) respectively, we have
$$
    u \wedge v = \Biggl(
    \left|\begin{array}{cc} 
        u_2 & v_2 \\ 
        u_3 & v_3
    \end{array}\right|,
    \left|\begin{array}{cc} 
        u_3 & v_3 \\ 
        u_1 & v_1
    \end{array}\right|,
    \left|\begin{array}{cc} 
        u_1 & v_1 \\ 
        u_2 & v_2
    \end{array}\right|
    \Biggr)
$$

\par
\textbf{Proposition 1.1 (vector product)}\\
1. (anticommutativity) $u \wedge v = - v \wedge u$.\\
2. $(au + bv) \wedge w = au \wedge w + bv \wedge w$.\\
3. $u \wedge v = 0$ if and only if $u$ and $v$ are linearly dependent.\\
4. $(u \wedge v) \cdot u = 0$, $(u \wedge v) \cdot v = 0$.\\

\par
\textbf{Proposition 1.2 (vector product)}\\
Given arbitrary vectors $u, v, x, y \in \mathbb{R}^3$,
$(u \wedge v) \cdot (x \wedge y) = 
\left|\begin{array}{cc} 
    u \cdot x & v \cdot x \\ 
    u \cdot y & v \cdot y
\end{array}\right|$.\\
\textbf{\textit{Proof.}}\\
$$
\begin{aligned}
    (u \wedge v) \cdot (x \wedge y) &= 
    \Biggl(
    \left|\begin{array}{cc} 
        u_2 & v_2 \\ 
        u_3 & v_3
    \end{array}\right|,
    \left|\begin{array}{cc} 
        u_3 & v_3 \\ 
        u_1 & v_1
    \end{array}\right|,
    \left|\begin{array}{cc} 
        u_1 & v_1 \\ 
        u_2 & v_2
    \end{array}\right|
    \Biggr)
    \cdot
    \Biggl(
    \left|\begin{array}{cc} 
        x_2 & y_2 \\ 
        x_3 & y_3
    \end{array}\right|,
    \left|\begin{array}{cc} 
        x_3 & y_3 \\ 
        x_1 & y_1
    \end{array}\right|,
    \left|\begin{array}{cc} 
        x_1 & y_1 \\ 
        x_2 & y_2
    \end{array}\right|
    \Biggr)\\
    &=
    (u_2v_3-u_3v_2)(x_2y_3-x_3y_2) +
    (u_3v_1-u_1v_3)(x_3y_1-x_1y_3) +
    (u_1v_2-u_2v_1)(x_1y_2-x_2y_1)\\
    &=
    (u_2v_3x_2y_3 + u_3v_2x_3y_2 + u_3v_1x_3y_1 + u_1v_3x_1y_3 + u_1v_2x_1y_2 + u_2v_1x_2y_1)\\
    &-(u_3v_2x_2y_3 + u_2v_3x_3y_2 + u_1v_3x_3y_1 + u_3v_1x_1y_3 + u_2v_1x_1y_2 + u_1v_2x_2y_1)\\
    &= 
    (u_1x_1 + u_2x_2 + u_3x_3)(v_1y_1 + v_2y_2 + v_3y_3)-(u_1y_1 + u_2y_2 + u_3y_3)(v_1x_1 + v_2x_2 + v_3x_3)\\
    &=
    \left|\begin{array}{cc} 
        u \cdot x & v \cdot x \\ 
        u \cdot y & v \cdot y
    \end{array}\right| \quad \qedsymbol\\
\end{aligned}
$$

\par
It follows immediately,
$$
    |u \wedge v|^2 = (u \wedge v)\cdot(u \wedge v) = |u|^2|v|^2(1 - \cos^2 \langle u,v\rangle) = |A|^2
$$

\par 
\textbf{Proposition 1.3 (vector product)}\\
Given arbitrary vectors $u,v,w \in \mathbb{R}^3$, $(u \wedge v) \wedge w = (u \cdot w) \wedge v - (v \cdot w) \wedge u$.\\
\textbf{\textit{Proof.}}\\
First we observe that both sides are linear in $u,v,w$, thus the identity will be true if it holds for
all basis vectors $e_1, e_2, e_3$. This last verification is straightforward, for example,
$$
    (e_1 \wedge e_2) \wedge e_3 = e_3 \wedge e_3 = 0 = (e_1 \cdot e_3) \wedge e_2 - (e_2 \cdot e_3) \wedge e_1
$$

\par
\textbf{Proposition 1.4 (vector product)}\\
Let $u, v$ be parametrized differentiable curves in $\mathbb{R}^3$, then
$$
    \frac{d}{dt}(u(t) \wedge v(t)) = u'(t) \wedge v(t) + u(t) \wedge v'(t)
$$
\textbf{\textit{Proof.}}\\
Suppose $u(t) = (x_1(t), y_1(t), z_1(t))$, $v(t) = (x_2(t), y_2(t), z_2(t))$.
Then
$$
\begin{aligned}
    u(t) \wedge v(t) 
    &= 
    \Biggl(
    \left|\begin{array}{cc} 
        u_2(t) & v_2(t) \\ 
        u_3(t) & v_3(t)
    \end{array}\right|,
    \left|\begin{array}{cc} 
        u_3(t) & v_3(t) \\ 
        u_1(t) & v_1(t)
    \end{array}\right|,
    \left|\begin{array}{cc} 
        u_1(t) & v_1(t) \\ 
        u_2(t) & v_2(t)
    \end{array}\right|
    \Biggr)\\
    &=
    (u_2(t)v_3(t)-u_3(t)v_2(t), u_3(t)v_1(t)-u_1(t)v_3(t), u_1(t)v_2(t)-u_2(t)v_1(t))\\
    &=
    (u_2(t)v_3(t), u_3(t)v_1(t), u_1(t)v_2(t)) - (u_3(t)v_2(t), u_1(t)v_3(t), u_2(t)v_1(t))
\end{aligned}
$$
The following verification is easy. \quad \qedsymbol

\par
\textbf{Definition 1.5 (curvature)}\\
Let $\alpha: I \to \mathbb R^3$ be a regular curve parametrized by arc length. Given $s \in I$, $k(s) = |\alpha''(s)|$ is
called the \textbf{curvature} of $\alpha$ at $s$.

\par
If $\alpha$ is a straight line, $\alpha(s) = us + v$, then $k \equiv 0$. It's also true conversely.

\par
\textbf{Definition 1.6 (tangent vector)}\\
Let $\alpha: I \to \mathbb R^3$ be a regular curve parametrized by arc length. The tangent vector at $s \in I$ is defined
by $t(s) = \alpha'(s)$.

\par
\textbf{Definition 1.7 (normal vector)}\\
Let $\alpha: I \to \mathbb R^3$ be a regular curve parametrized by arc length, at points where $k(s) \neq 0$,
a unit vector $n(s)$ is in the direction $\alpha''(s)$ is well defined by the equation
$\alpha''(s) = k(s)n(s)$.

\par
\textbf{Proposition 1.5 (normal vector)}\\
Let $\alpha: I \to \mathbb R^3$ be a regular curve parametrized by arc length, for any $s \in I$, $\alpha''(s)$ is normal to $\alpha'(s)$.\\
\textbf{\textit{Proof.}}\\
Since we have $\alpha'(s) \cdot \alpha'(s) = 1$, take derivatives on both sides of the equation and it follows
$$
    \quad \alpha''(s) \cdot \alpha'(s) = 0 \quad \qedsymbol
$$

\par
\textbf{Definition 1.8 (osculating plane)}\\
The plane determined by tangent vector $t(s)$ and normal vector $n(s)$ is called the osculating plane at $s$.

\par
\textbf{Remark.}\\
At points where $k(s) = 0$, the normal vector (and therefore the osculating plane) is not defined. To proceed
with the local analysis of curves, we need, in an essential way, the osculating plane. It is therefore convenient
to say that $s \in I$ is a singular point of order 1 if $\alpha''(s)=0$.
\textbf{In what follows, we shall restrict ourselves to curves parametrized by arc length without singular points
of order 1}.

\par
\textbf{Definition 1.9 (binormal vector)}\\
The unit vector $b(s) = t(s) \wedge n(s)$ is normal to the osculating plane and will be called
the binormal vector at $s$. Since $b(s)$ is a unit vector, the length $|b'(s)|$ measures the rate of
change of the neighboring osculating planes with the osculating plane at $s$. That is, $|b'(s)|$ measures
how rapidly the curve pulls away from the osculating plane at $s$, in a neighborhood of $s$.\\

\par
To compute $b'(s)$ we observe that, on the one hand, $b'(s)$ is normal to $b(s)$ and that, on the other hand,
$$
    b'(s) = t'(s) \wedge n(s) + t(s) \wedge n'(s) = \alpha''(s) \wedge \frac{\alpha''(s)}{k(s)} + t(s) \wedge n'(s) = t(s) \wedge n'(s)
$$
that is, $b'(s)$ is normal to $t(s)$. Also, we know that $b(s) \cdot b(s) = 1$, which implies $b'(s) \cdot b(s) = 0$, so $b'(s)$ is parallel to $n(s)$.
Hence we can write $b'(s) = \tau(s)n(s)$, where $\tau: I \to \mathbb{R}$.

\par
\textbf{Definition 1.10 (tortion)}\\
Let $\alpha: I \to \mathbb R^3$ be a regular curve parametrized by arc length $s$ such that $\alpha''(s) \neq 0$, $s \in I$.
The number $\tau(s)$ defined by $b'(s) = \tau(s)n(s)$ is called the tortion of $\alpha$ at $s$.

\par
\textbf{Proposition 1.6 (plane curve)}\\
Let $\alpha: I \to \mathbb R^3$ be a regular curve parametrized by arc length, assume that for all $s \in I$, $k(s) \neq 0$, then
$\alpha$ is a plane curve if and only if $\tau(s) \equiv 0$.

\par
We have already known that $t'(s) = k(s)n(s)$, $b'(s) = \tau(s)n(s)$, it remains to calculate $n'(s)$.
$$
\begin{aligned}
    n'(s) &= (b(s) \wedge t(s))' = b'(s) \wedge t(s) + b(s) \wedge t'(s)\\
    &= (\tau(s)n(s)) \wedge t(s) + b(s) \wedge (k(s)n(s))\\
    &= -\tau(s)b(s) - k(s)t(s)
\end{aligned}
$$

\par
\textbf{Definition 1.11 (Frenet formulas)}\\
$$
    t' = kn
$$
$$
    n' = -\tau b - kt
$$
$$
    b' = \tau n
$$
We can write it in the form of matrix:
$$
    \left(\begin{array}{c}
        t'\\n'\\b'
    \end{array}
    \right)
    =
    \left(\begin{array}{ccc} 
     & k & \\ 
    -k & & -\tau \\
     & \tau &
    \end{array}\right)
    \left(\begin{array}{c}
        t\\n\\b
    \end{array}
    \right)
$$
The $tb$ plane is called the rectifying plane, and the $nb$ plane the normal plane.

\par
\textbf{Theorem 1.1 (fundamental theorem of the local theory of curves)}\\
Given differentiable function $k(s)>0$ and $\tau(s)$, $s \in I$, there exists a regular
parametrized curve $\alpha: I \to \mathbb{R}^3$ such that $s$ is the arc length, $k(s)$ is
the curvature, and $\tau(s)$ is the tortion of $\alpha$. Moreover, any other curve 
$\overline{\alpha}$, satisfying the same conditions, differs from $\alpha$ by a rigid motion.
That is, there exists an orthonormal linear map $\rho$ of $\mathbb{R}^3$, with positive
determinant, and a vector $c$ such that $\overline{\alpha} = \rho \alpha + c$.

\end{document}