\documentclass{article}

\usepackage{changepage,aligned-overset,amsfonts,amssymb,amsthm,enumerate,geometry,mathrsfs}

\geometry{a4paper, scale = 0.8}

\author{Zheng Xie}
\title{Surfaces}
\date{May 28, 2018}
    
\begin{document}
\maketitle
    
\setlength\parindent{0em}   % cancel all indent
\setlength\parskip{1.0\baselineskip} % set skip between paragraphs

\par
Roughly speaking, a regualr surface in $\mathbb{R}^3$ is obtained by taking pieces of a plane,
deforming them, and arranging them in such a way that the resulting figure has no sharp points, edges, or self-intersections
and so that it makes sense to speak of a tangent plane at points of the figure.\\
The idea is to define a set that is, in a certain sense, two-dimensional and that also is smooth
enough so that the usual notions of calculus can be extended to it.
    
\par
\textbf{Definition 2.1 (Regular Surface)}\\
A subset $S \subset \mathbb R^3$ is a regular surface if for each $p \in S$, there exists a neighborhood
$V \subset \mathbb R^3$, an open set $U \subset \mathbb R^2$ and an onto map $\mathcal{X}:U \to V \cap S$ such that\\
(1) $\mathcal{X}$ is differentiable, i.e. if we write\\
$$
    \mathcal{X}(u,v) = (x(u,v), y(u,v), z(u,v)), (u,v) \in U
$$
Then the functions $x(u,v), y(u,v), z(u,v)$ have continuous partial derivatives of all orders in U.\\
(2) $\mathcal{X}$ is a homeomorphism, since $\mathcal{X}$ is continuous by condition (1), this means that 
$\mathcal{X}^{-1}: V \cap S \to U$ is continuous.\\
(3) (regularity condition) For each $q \in U$, the differential $d\mathcal{X}_q: \mathbb{R}^2 \to \mathbb{R}^3$
is one-to-one.

\par
The mapping $\mathcal{X}$ is called a parametrization or system of (local) coordinates
in a neighborhood of $p$. The neighborhood $V \cap S$ of $p$ in $S$ is called a coordinate neighborhood.

\par
To give condition (3) a more familiar form, let us compute the matrix of the linear map $d\mathcal{X}_q$ in the
canonical bases $e_1 = (1,0)$, $e_2 = (0,1)$ of $\mathbb{R}^2$ with coordinates $(u,v)$ and
$f_1 = (1,0,0)$, $f_2 = (0,1,0)$, $f_3 = (0,0,1)$ of $\mathbb{R}^3$, with coordinates $(x,y,z)$.\\
Let $q = (u_0, v_0)$, the vector $e_1$ is tangent to the curve $\alpha: \mathbb{R} \to U \subset \mathbb{R}^2, u \mapsto (u,v_0)$
whose image under $\mathcal{X}$ is
the curve
$$
    \beta: \mathbb{R} \to \mathbb{R}^3, \quad u \mapsto (x(u, v_0), y(u, v_0), z(u, v_0))
$$
This image curve (called the coordinate curve $v = v_0$) lies on $S$ and has the tangent vector at $\mathcal{X}(q)$,
which is defined by
$$
    (\frac{\partial x}{\partial u}(u_0,v_0), \frac{\partial y}{\partial u}(u_0,v_0), \frac{\partial z}{\partial u}(u_0,v_0)) = \frac{\partial \mathcal{X}}{\partial u}(u_0,v_0)
$$\\
By the definition of differential,
$$
    d\mathcal{X}_q(e_1) = \frac{\partial \mathcal{X}}{\partial u}(u_0, v_0)
$$
$$
    d\mathcal{X}_q(e_2) = \frac{\partial \mathcal{X}}{\partial v}(u_0, v_0)
$$\\
Thus, the matrix of the linear map $d\mathcal{X}_q$ in the referred basis is
$$
    d\mathcal{X}_q = 
    \left(\begin{array}{cc} 
        \dfrac{\partial x}{\partial u} & \dfrac{\partial x}{\partial v} \\\\
        \dfrac{\partial y}{\partial u} & \dfrac{\partial y}{\partial v} \\\\
        \dfrac{\partial z}{\partial u} & \dfrac{\partial z}{\partial v}
    \end{array}\right)_q
$$


\par
\textbf{Example 2.1 (unit sphere)}\\
The unit sphere
$$
    S^2 = \{(x,y,z) \in \mathbb{R}^3: x^2+y^2+z^2=1\}
$$
is a regular surface.\\
\textbf{\textit{Proof.}}\\
We first verify that the map $\mathcal{X}_1: U \subset \mathbb{R}^2 \to \mathbb{R}^3$ given by
$$
    \mathcal{X}_1(x,y) = (x, y, \sqrt{1-(x^2+y^2)}), (x,y) \in U
$$
is a parametrization of $S^2$. Observe that $\mathcal{X}_1(U)$ is the open part of $S^2$ above
the $xOy$ plane.\\
Since $x^2+y^2<1$, the function $\sqrt{1-(x^2+y^2)}$ has continuous partial derivatives of all orders.
Thus, $\mathcal{X}_1$ is differentiable and condition (1) holds.\\
To check condition (2), we observe that $\mathcal{X}_1$ is one-to-one and that $\mathcal{X}_1^{-1}$ is
the restriction of the continuous projection $\pi(x,y,z) = (x,y)$ to the set $\mathcal{X}_1(U)$. Thus,
$\mathcal{X}_1^{-1}$ is continuous in $\mathcal{X}_1(U)$.\\
To check condition (3), note that
$$
    d{\mathcal{X}_{1}} = 
    \left(\begin{array}{cc} 
        1 & 0 \\\\
        0 & 1 \\\\
        \dfrac{-x}{\sqrt{1-(x^2+y^2)}} & \dfrac{-y}{\sqrt{1-(x^2+y^2)}}
    \end{array}\right)
$$
It's easy to see that $d\mathcal{X}_1$ is one-to-one.\\
Similarly we can verify the above conditions for other parametrizations like
$$
    \mathcal{X}_2(x,z) = (x, \sqrt{1-(x^2+z^2)}, z)
$$
$$
    \mathcal{X}_3(y,z) = (\sqrt{1-(y^2+z^2)}, y, z)
$$

\par
\textbf{Proposition 2.1 (a simple property of regular surface)}\\
Let $S \subset \mathbb{R}^2$ be a regular surface, then $S$ is a locally graph, i.e.\\
$\forall p \in S$, $\exists V$ an open subset of $S$ containing $p$ such that
$$
    V = \{(x,y,f(x,y)): (x,y) \in U \subset \mathbb{R}^2\}
$$
or
$$
    V = \{(x,f(x,z),z): (x,z) \in U \subset \mathbb{R}^2\}
$$
or
$$
    V = \{(f(y,z),y,z): (y,z) \in U \subset \mathbb{R}^2\}
$$
where $U$ is an open set, $f: \mathbb{R}^2 \to \mathbb{R}$ is a differentiable map.\\
\textbf{\textit{Proof.}}\\
Take any $p \in S$, there exists a differentiable map $\mathcal{X}: U \subset \mathbb{R}^2 \to V \cap S \subset \mathbb{R}^3$
such that $\mathcal{X}(q)=p$, where $U \subset \mathbb{R}^2,V \subset \mathbb{R}^3$ are open sets. Moreover,
$$
    d\mathcal{X}_q = 
    \left(\begin{array}{cc} 
        \dfrac{\partial x}{\partial u} & \dfrac{\partial x}{\partial v} \\\\
        \dfrac{\partial y}{\partial u} & \dfrac{\partial y}{\partial v} \\\\
        \dfrac{\partial z}{\partial u} & \dfrac{\partial z}{\partial v}
    \end{array}\right)_q
$$
has rank 2.\\
Without loss of generality, we can first assume that
$$
    (\dfrac{\partial x}{\partial u}, \dfrac{\partial x}{\partial v}) \text{ and } (\dfrac{\partial y}{\partial u}, \dfrac{\partial y}{\partial v})
$$
are linearly independent at point $q$. That is,
$$
\left(\begin{array}{cc} 
    \dfrac{\partial x}{\partial u} & \dfrac{\partial x}{\partial v} \\\\
    \dfrac{\partial y}{\partial u} & \dfrac{\partial y}{\partial v}
\end{array}\right)_q
$$
is invertible.\\
Consider projection $\pi: \mathbb{R}^3 \to \mathbb{R}^2$, $\pi (x,y,z) = (x,y)$. It's easy to
verify that $\pi$ is a differentiable map. Then
$$
    \pi \circ \mathcal{X}: U \to W=\pi(\mathcal{X}(U))
$$
is also a differentiable map, and
$$
    d(\pi \circ \mathcal{X})_q = 
    \left(\begin{array}{cc} 
        \dfrac{\partial x}{\partial u} & \dfrac{\partial x}{\partial v} \\\\
        \dfrac{\partial y}{\partial u} & \dfrac{\partial y}{\partial v}
    \end{array}\right)_q
$$
is invertible.\\
Now we can apply inverse function theorem to $\pi \circ \mathcal{X}$ at $q$:\\
Hence, there exists $U_1 \subset U$, let $W_1 = \pi(X(U_1))$, then $\pi \circ \mathcal{X}: U_1 \to W_1$
is a differentiable map and 
$$
    (\pi \circ \mathcal{X})^{-1}: W_1 \to U_1
$$
exists, which is also differentiable(hence continuous, and it follows that $W_1$ is an open set).\\
Consider another map $\mathcal{Y}$
$$
    \mathcal{Y} = \mathcal{X} \circ (\pi \circ \mathcal{X})^{-1}: W_1 \to \mathcal{X}(U_1), (x,y) \mapsto (x,y,f(x,y))
$$
Therefore, given $p \in S$, we can find an open subset $V=\mathcal{X}(U_1) \subset \mathbb{R}^3$ containing $p$ and an open subset $U=W_1 \subset \mathbb{R}^2$
such that
$$
    V = \{(x,y,f(x,y)): (x,y) \in U \subset \mathbb{R}^2\}
$$


\par
\textbf{Definition 2.2 (Principle Curvature)}\\
Let $S \subset \mathbb R^3$ be a regular surface.

\end{document}