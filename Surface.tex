\documentclass{article}

\usepackage{changepage,aligned-overset,amsfonts,amssymb,amsthm,enumerate,geometry,mathrsfs}

\geometry{a4paper, scale = 0.8}

\author{Zheng Xie}
\title{Surfaces}
\date{May 28, 2018}
    
\begin{document}
\maketitle
    
\setlength\parindent{0em}   % cancel all indent
\setlength\parskip{1.0\baselineskip} % set skip between paragraphs

\par
Roughly speaking, a regular surface in $\mathbb{R}^3$ is obtained by taking pieces of a plane,
deforming them, and arranging them in such a way that the resulting figure has no sharp points, edges, or self-intersections
and so that it makes sense to speak of a tangent plane at points of the figure.\\
The idea is to define a set that is, in a certain sense, two-dimensional and that also is smooth
enough so that the usual notions of calculus can be extended to it.
    
\par
\textbf{Definition 2.1 (Regular Surface)}\\
A subset $S \subset \mathbb R^3$ is a regular surface if for each $p \in S$, there exists a neighborhood
$V \subset \mathbb R^3$, an open set $U \subset \mathbb R^2$ and an onto map $\mathcal{X}:U \to V \cap S$ such that\\
(1) $\mathcal{X}$ is differentiable, i.e. if we write\\
$$
    \mathcal{X}(u,v) = (x(u,v), y(u,v), z(u,v)), (u,v) \in U
$$
Then the functions $x(u,v), y(u,v), z(u,v)$ have continuous partial derivatives of all orders in U.\\
(2) $\mathcal{X}$ is a homeomorphism, since $\mathcal{X}$ is continuous by condition (1), this means that 
$\mathcal{X}^{-1}: V \cap S \to U$ is continuous.\\
(3) (regularity condition) For each $q \in U$, the differential $d\mathcal{X}_q: \mathbb{R}^2 \to \mathbb{R}^3$
is one-to-one.

\par
The mapping $\mathcal{X}$ is called a parametrization or system of (local) coordinates
in a neighborhood of $p$. The neighborhood $V \cap S$ of $p$ in $S$ is called a coordinate neighborhood.

\par
To give condition (3) a more familiar form, let us compute the matrix of the linear map $d\mathcal{X}_q$ in the
canonical bases $e_1 = (1,0)$, $e_2 = (0,1)$ of $\mathbb{R}^2$ with coordinates $(u,v)$ and
$f_1 = (1,0,0)$, $f_2 = (0,1,0)$, $f_3 = (0,0,1)$ of $\mathbb{R}^3$, with coordinates $(x,y,z)$.\\
Let $q = (u_0, v_0)$, the vector $e_1$ is tangent to the curve $\alpha: \mathbb{R} \to U \subset \mathbb{R}^2, u \mapsto (u,v_0)$
whose image under $\mathcal{X}$ is
the curve
$$
    \beta: \mathbb{R} \to \mathbb{R}^3, \quad u \mapsto (x(u, v_0), y(u, v_0), z(u, v_0))
$$
This image curve (called the coordinate curve $v = v_0$) lies on $S$ and has the tangent vector at $\mathcal{X}(q)$,
which is defined by
$$
    (\frac{\partial x}{\partial u}(u_0,v_0), \frac{\partial y}{\partial u}(u_0,v_0), \frac{\partial z}{\partial u}(u_0,v_0)) = \frac{\partial \mathcal{X}}{\partial u}(u_0,v_0)
$$\\
By the definition of differential,
$$
    d\mathcal{X}_q(e_1) = \frac{\partial \mathcal{X}}{\partial u}(u_0, v_0)
$$
$$
    d\mathcal{X}_q(e_2) = \frac{\partial \mathcal{X}}{\partial v}(u_0, v_0)
$$\\
Thus, the matrix of the linear map $d\mathcal{X}_q$ in the referred basis is
$$
    d\mathcal{X}_q = 
    \left(\begin{array}{cc} 
        \dfrac{\partial x}{\partial u} & \dfrac{\partial x}{\partial v} \\\\
        \dfrac{\partial y}{\partial u} & \dfrac{\partial y}{\partial v} \\\\
        \dfrac{\partial z}{\partial u} & \dfrac{\partial z}{\partial v}
    \end{array}\right)_q
$$


\par
\textbf{Example 2.1 (unit sphere)}\\
The unit sphere
$$
    S^2 = \{(x,y,z) \in \mathbb{R}^3: x^2+y^2+z^2=1\}
$$
is a regular surface.\\
\textbf{\textit{Proof.}}\\
We first verify that the map $\mathcal{X}_1: U \subset \mathbb{R}^2 \to \mathbb{R}^3$ given by
$$
    \mathcal{X}_1(x,y) = (x, y, \sqrt{1-(x^2+y^2)}), (x,y) \in U
$$
is a parametrization of $S^2$. Observe that $\mathcal{X}_1(U)$ is the open part of $S^2$ above
the $xOy$ plane.\\
Since $x^2+y^2<1$, the function $\sqrt{1-(x^2+y^2)}$ has continuous partial derivatives of all orders.
Thus, $\mathcal{X}_1$ is differentiable and condition (1) holds.\\
To check condition (2), we observe that $\mathcal{X}_1$ is one-to-one and that $\mathcal{X}_1^{-1}$ is
the restriction of the continuous projection $\pi(x,y,z) = (x,y)$ to the set $\mathcal{X}_1(U)$. Thus,
$\mathcal{X}_1^{-1}$ is continuous in $\mathcal{X}_1(U)$.\\
To check condition (3), note that
$$
    d{\mathcal{X}_{1}} = 
    \left(\begin{array}{cc} 
        1 & 0 \\\\
        0 & 1 \\\\
        \dfrac{-x}{\sqrt{1-(x^2+y^2)}} & \dfrac{-y}{\sqrt{1-(x^2+y^2)}}
    \end{array}\right)
$$
It's easy to see that $d\mathcal{X}_1$ is one-to-one.\\
Similarly we can verify the above conditions for other parametrizations like
$$
    \mathcal{X}_2(x,z) = (x, \sqrt{1-(x^2+z^2)}, z)
$$
$$
    \quad \quad \mathcal{X}_3(y,z) = (\sqrt{1-(y^2+z^2)}, y, z) \qed
$$

\par
\textbf{Proposition 2.1}\\
If $f: U \subset \mathbb{R}^2 \to \mathbb{R}^3$ is a differentiable function in an open set $U$, then
the graph of $f$, that is, the subset of $\mathbb{R}^3$ given by $(x, y, f(x,y))$ for $(x,y) \in U$, is
a regular surface.\\
\textbf{\textit{Proof.}}\\
It suffices to show that the map $\mathcal{X}: U \to \mathbb{R}^3$ given by
$$
    \mathcal{X}(u,v) = (u,v,f(u,v))
$$
is a parametrization of the graph whose coordinate neighborhood covers every point
of the graph.\\
It's clearly that $\mathcal{X}$ is differentiable.\\
Also note that
$$
    \mathcal{X}^{-1} = \pi |_{G}
$$
where $\pi: \mathbb{R}^3 \to \mathbb{R}^2$ is bounded and thus continuous, $G=\{(x,y,f(x,y)): (x,y) \in U\}$. So $\mathcal{X}^{-1}$
is continuous and thus $\mathcal{X}$ is a homeomorphism.\\
Finally, since
$$
    d{\mathcal{X}} = 
    \left(\begin{array}{cc} 
        1 & 0 \\\\
        0 & 1 \\\\
        \dfrac{df}{du} & \dfrac{df}{dv}
    \end{array}\right)
$$
has rank 2 at any point $(u,v) \in U$, it follows that $\mathcal{X}$ is a parametrization.\quad \qedsymbol

\par
\textbf{Definition 2.2 (critical and regular point)}\\
Given a differentiable map $F: U \subset \mathbb{R}^n \to \mathbb{R}^m$ defined in an open set $U$ of $\mathbb{R}^n$ we
say that $p \in U$ is a critical point of $F$ if the differential $dF_p: \mathbb{R}^n \to \mathbb{R}^m$
is not a surjective (onto) mapping. The image $F(p) \in \mathbb{R}^m$ of a critical point is called a critical value of
$F$. A point of $\mathbb{R}^m$ which is not a critical value is called a regular value of $F$.


\par
\textbf{Proposition 2.2}\\
Let $f: U \subset \mathbb{R}^3 \to \mathbb{R}$ be a differentiable function, $U$ is an open subset of $\mathbb{R}^3$, $a$ is a regular
value of $f$, then $f^{-1}(a)$ is a regular surface.\\
\textbf{\textit{Proof.}}\\
Pick any $p \in f^{-1}(a)$, since $a$ is a regular value, we can assume without loss of generality that $\frac{\partial f}{\partial z} \neq 0$.\\
Then define $F: U \subset \mathbb{R}^3 \to \mathbb{R}^3$ by
$$
    F(x,y,z) = (x, y, f(x,y,z))
$$
Note that
$$
    dF_p = 
    \left(\begin{array}{ccc}
        1 & 0 & 0 \\\\
        0 & 1 & 0 \\\\
        \dfrac{\partial f}{\partial x} & \dfrac{\partial f}{\partial y} & \dfrac{\partial f}{\partial z}
    \end{array}\right)_p
$$
is invertible. So by inverse function theorem, there exists a neighborhood $V_p \subset \mathbb{R}^3$ of $p$
and $W_p = F(V_p) \subset \mathbb{R}^3$ such that
$F: V_p \to W_p$ is invertible and $F^{-1}: W_p \to V_p$ is differentiable. Note that $F^{-1}(x,y,a)=(x,y,z)$,
hence $F^{-1}: W_p \to V_p$ can be written as
$$
    F^{-1}(x,y,a) = (x,y,h(x,y))
$$
where $z=h(x,y)$ is a differentiable function here.\\
Therefore, $V_p = \{(x,y,h(x,y)): (x,y) \in W_p\}$ is the graph of $h(x,y)$ and by $Proposition 2.1$ we know that
$V_p$ is a regular surface. Since $p$ is arbitrary in $f^{-1}(a)$, $f^{-1}(a)$ is a regular surface. \qedsymbol

\par
\textbf{Proposition 2.3}\\
Let $S \subset \mathbb{R}^3$ be a regular surface and $p \in S$, then there exists a neighborhood $V$ of $p \in S$
such that $V$ is the graph of a differentiable function which has one of the following three forms:
$z=f(x,y), y=g(x,z), x=h(y,z)$.\\
\textbf{\textit{Proof.}}\\
Since $S$ is a regular surface, we know that there exists an open set $U \subset \mathbb{R}^2$, $V=\mathcal{X}(U) \subset \mathbb{R}^3$
and a diffeomorphism $\mathcal{X}: U \to V$ such that
$$
    d\mathcal{X}_q = 
    \left(
    \begin{array}{cc}
        \dfrac{\partial x}{\partial u} & \dfrac{\partial x}{\partial v}\\\\
        \dfrac{\partial y}{\partial u} & \dfrac{\partial y}{\partial v}\\\\
        \dfrac{\partial z}{\partial u} & \dfrac{\partial z}{\partial v}
    \end{array}
    \right)_q
$$
has rank $2$, where $\mathcal{X}(q)=p, q \in U$
Without loss of generality, we can assume that
$$
    (\frac{\partial x}{\partial u}, \frac{\partial x}{\partial v}), (\frac{\partial y}{\partial u}, \frac{\partial y}{\partial v})
$$
are linearly independent at point $q$.\\
Consider projection $\pi: S \subset \mathbb{R}^3 \to \mathbb{R}^2$, then $\pi \circ \mathcal{X}: U \to W=\pi(\mathcal{X}(U)) \subset \mathbb{R}^2$ is differentiable and
$$
    d(\pi \circ \mathcal{X}) = 
    \left(
    \begin{array}{cc}
        \dfrac{\partial x}{\partial u} & \dfrac{\partial x}{\partial v}\\\\
        \dfrac{\partial y}{\partial u} & \dfrac{\partial y}{\partial v}
    \end{array}
    \right)
$$
whose determinant is non-zero at point $q$ by the assumption above.\\
Therefore by inverse function theorem, we can shrink $U$ and claim that $(\pi \circ \mathcal{X})^{-1}: W \to U$ exists and
is differentiable.\\
Now let $F = \mathcal{X} \circ (\pi \circ \mathcal{X})^{-1}$, then $F: W \subset \mathbb{R}^2 \to V \subset \mathbb{R}^3, (x,y) \mapsto (x,y,z)$ is
differentiable, so there exists a differentiable function $f: W \to \mathbb{R}$ such that $z=f(x,y)$ for each $(x,y) \in W$.\\
Moreover, since $\mathcal{X}$ is a homeomorphism, $V=\mathcal{X}(U)$ is a (open) neighborhood of $p$, therefore,
$$
    V = \{(x,y,f(x,y)): (x,y) \in W\} \qed
$$

\par
\textbf{Remark.}\\
The condition that $\mathcal{X}$ is a homeomorphism is required, otherwise we may not find an open neighborhood of $p$.

\par
\textbf{Proposition 2.4}\\
Let $p$ be a point of a regular surface $S$ and let $\mathcal{X}: U \subset \mathbb{R}^2 \to \mathbb{R}^3$ be a map with $p \in \mathcal{X}(U) \subset S$
such that conditions 1(differentiable) and 3(regularity condition) hold. Assume that $\mathcal{X}$ is one-to-one, then $\mathcal{X}^{-1}$ is continuous.\\
\textbf{\textit{Proof.}}\\
Assume that $\mathcal{X}(u,v)=(x(u,v),y(u,v),z(u,v))$, then by condition 1 and condition3, $x, y, z$ are differentiable and
$$
    d\mathcal{X} = 
    \left(
        \begin{array}{cc}
            \dfrac{\partial x}{\partial u} & \dfrac{\partial x}{\partial v}\\\\
            \dfrac{\partial y}{\partial u} & \dfrac{\partial y}{\partial v}\\\\
            \dfrac{\partial z}{\partial u} & \dfrac{\partial z}{\partial v}
        \end{array}
    \right)
$$
has rank 2 in $U$.\\
Similarly we can assume without loss of generality that $\dfrac{\partial(x,y)}{\partial(u,v)} \neq 0$, then consider 
$$
    \pi: \mathbb{R}^3 \to \mathbb{R}^2, (x,y,z) \mapsto (x,y)
$$
Note that $\pi \circ \mathcal{X}: U \to \pi(\mathcal{X}(U)) \subset \mathbb{R}^2$ is differentiable and
$$
    d(\pi \circ \mathcal{X}) = 
    \left(
        \begin{array}{cc}
            \dfrac{\partial x}{\partial u} & \dfrac{\partial x}{\partial v}\\\\
            \dfrac{\partial y}{\partial u} & \dfrac{\partial y}{\partial v}
        \end{array}
    \right)
$$
which is invertible in $q$. Therefore, by inverse function theorem, the inverse
$$
    (\pi \circ \mathcal{X})^{-1}: \pi(\mathcal{X}(U)) \to U
$$
exists (with shrunk $U$) and is differentiable, hence continuous.
Therefore, $\mathcal{X}^{-1} = (\pi \circ \mathcal{X})^{-1} \circ \pi$, as a composition of two
continuous funciton, is also continuous. \qedsymbol

\par
\textbf{Proposition 2.5 (change of parameters)}\\
Let $p$ be a point of a regular surface $S$, and let $\mathcal{X}: U \subset \mathbb{R}^2 \to S, \mathcal{Y}: V \subset \mathbb{R}^2 \to S$
be two parametrizations of $S$ such that $p \in \mathcal{X}(U) \cap \mathcal{Y}(V) = W$. Then
$$
    h = \mathcal{X}^{-1} \circ \mathcal{Y}: \mathcal{Y}^{-1}(W) \to \mathcal{X}^{-1}(W)
$$
is a diffeomorphism.\\
\textbf{\textit{Proof.}}\\
It's clear that $h$ as a composition of two homeomorphisms is a homeomorphism.\\
Let $r \in \mathcal{Y}^{-1}(W)$ and set $q = h(r)$. Since $\mathcal{X}(u,v) = (x(u,v), y(u,v), z(u,v))$ is a parametrization,
we can assume, by renaming the axes if necessary, that
$$
    \frac{\partial (x,y)}{\partial (u,v)}(q) \neq 0
$$
Then, we extend $\mathcal{X}$ to a map $F: U \times \mathbb{R} \to \mathbb{R}^3$ defined by
$$
    F(u,v,t) = (x(u,v), y(u,v), z(u,v)+t), (u,v) \in U, t \in \mathbb{R}
$$
It's clear that $F$ is differentiable and that the restriction $F|_{U \times \{0\}} = \mathcal{X}$, note that
$$
    det(dF_q) = \frac{\partial (x,y)}{\partial (u,v)}(q) \neq 0
$$
So we can apply the inverse function theorem here:\\
There exists a neighborhood of $\mathcal{X}(q)$$M \subset W$ and 

\par
\textbf{Definition 2.3}\\
Let $f: V \subset S \to \mathbb{R}$ be a function defined in an open subset $V$ of a regular surface $S$. Then $f$ is said to be
differentiable at $p \in V$ if, for some parametrization $\mathcal{X}: U \subset \mathbb{R}^2 \to S$ with $p \in \mathcal{X}(U) \subset V$, the 
composition $f \circ \mathcal{X}: U \subset \mathbb{R}^2$ is differentiable at $\mathcal{X}^{-1}(p)$. $f$ is differentiable in $V$ if it is differentiable
at all points of $V$.


\end{document}