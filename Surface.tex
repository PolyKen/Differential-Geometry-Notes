\documentclass{article}

\usepackage{changepage,aligned-overset,amsfonts}

\author{Zheng Xie}
\title{Surfaces}
\date{May 28, 2018}
    
\begin{document}
\maketitle
    
\setlength\parindent{0em}   % cancel all indent
\setlength\parskip{1.0\baselineskip} % set skip between paragraphs
    
\par
\textbf{Definition 2.1 (Regular Surface)}\\
A subset $S \subset \mathbb R^3$ is a regular surface if for each $p \in S$, there exists a neighborhood
$V \in \mathbb R^3$, an open set $U \in \mathbb R^2$ and an onto map $x:U \to V \cap S$ such that\\
(1)$x$ is differentiable, i.e. if $x(u,v) = (x_1(u,v),x_2(u,v),x_3(u,v))$, $(u,v) \in U$, then $x_i(u,v)$
have continuous partial derivatives of all orders in U.\\
(2)$x$ is a homeomorphism, i.e. $x^{-1}: V \cap S \to U$ is continuous.\\
(3)(regularity condition)\\For each $q \in U$, 

\par
\textbf{Definition 2.2 (Principle Curvature)}\\
Let $S \subset \mathbb R^3$ be a regular surface.

\end{document}