\documentclass{article}

\usepackage{changepage,aligned-overset,amsfonts,amssymb,amsthm,enumerate,geometry}
\usepackage{xeCJK}
\geometry{a4paper, scale = 0.8}
\setCJKmainfont{STSong}
\begin{document}
%\author{谢铮 15338200}
\title{Homework 3}
\date{Mar 22, 2019}
\maketitle

\setlength\parindent{0em}   % cancel all indent
\setlength\parskip{1.0\baselineskip} % set skip between paragraphs

\par
\textbf{2-3 Ex.3}\\
Show that the paraboloid $z=x^2+y^2$ is diffeomorphic to a plane.

\par
\textbf{\textit{Solution.}}\\
Consider map
$$
    \mathcal{X}(u,v)=(u,v,u^2+v^2)
$$
It's easy to see that $\mathcal{X}$ is differentiable, bijective, and 
$\frac{\partial{(u,v)}}{\partial{(u,v)}}=1$, so it suffices to show that
$\mathcal{X}^{-1}$ is continuous. Since $\mathcal{X}^{-1}$ can
be seen as a restriction of $\pi: \mathbb{R}^3 \to \mathbb{R}^2$ to 
$S=\{(x,y,z):z=x^2+y^2\}$, $\mathcal{X}^{-1}$ is also continuous. \qed

\par
\textbf{2-3 Ex.6}\\
Prove that the definition of a differentiable map between surfaces does
not depend on the parametrizations chosen.

\par
\textbf{\textit{Solution.}}\\
Suppose $\phi: S_1 \to S_2$ is a differentiable map where $S_1, S_2$ are
regular surfaces. By definition we know that given $p \in S_1$,
there exists open
sets $q \in U \subset \mathbb{R}^2$, 
$\bar{q} \in \bar{U} \subset \mathbb{R}^2$
and parametrizations $\mathcal{X}: U \to V \cap S_1$,
$\bar{\mathcal{X}}: \bar{U} \to V \cap S_2$
such that $p = \mathcal{X}(q), \phi(p) = \bar{\mathcal{X}}(\bar{q})$
and $f=\bar{\mathcal{X}}^{-1} \circ \phi \circ \mathcal{X}$
is differentiable at $q$.\\
Note that for $p \in S_1$ and $\phi(p) \in S_2$, we can find
another two parametrizations $\mathcal{Y}$ and $\bar{\mathcal{Y}}$
of $S_1$ at $p$ and $S_2$ at $\phi(p)$ respectively and
moreover,
$\mathcal{X} \circ \mathcal{Y}^{-1}$ and
$\bar{\mathcal{Y}} \circ \bar{\mathcal{X}}^{-1}$ are both
diffeomorphism. Therefore
$$
    g = \bar{\mathcal{Y}}^{-1} \circ \phi \circ \mathcal{Y}
$$
is also differentiable at $q$, which implies that the 
definition doesn't depend on the parametrizations chosen. \qed

\par
\textbf{2-3 Ex.10}\\
Let $C$ be a plane regular curve which lies in one side of
a straight line $r$ of the plane and meets $r$ at the points
$p,q$. What conditions should $C$ satisfy to ensure that the
rotation of $C$ about $r$ generates an extended regular
surface of revolution?

\par
\textbf{\textit{Solution.}}\\
For simplicity, we can assume that $C$ is parametrized by
$$
    \alpha: [0,1] \to C
$$
and $r$ is the rotation axis.
where $\alpha$ is smooth and injective(hence $C$ is simple).
$\alpha(t)=(\alpha_1(t), \alpha_2(t))$.
and $\alpha(0)=p, \alpha(1)=q$.\\
We have known that the surface of revolution denoted by $S$
is regular outside $p,q$ since $C$ is regular. Now assume
that $S$ is also regular at $p$ and $q$. We shall notice that
the tangent plane of $S$ at $p,q$, denoted by $T_p(S)$ and
$T_q(S)$ respectively, should stay invariant under rotation
$$
    R = 
    \left(
    \begin{array}{ccc}
        1 & 0 & 0\\
        0 & \cos(\theta) & -\sin(\theta)\\
        0 & \sin(\theta) & \cos(\theta)
    \end{array}
    \right)
$$
Therefore, the equation of $T_p(S)$ is given by
$$
    x = \alpha_1(0)
$$
Then let $\tilde{C}=S\cap \{z=0\}$, naturally we can also
find a parametrization of $\tilde{C}$ by
$$
    \tilde{\alpha}(t)
    =
    \begin{cases}
        (\alpha_1(t), \alpha_2(t)),\quad t \geq 0\\
        (\alpha_1(-t), -\alpha_2(-t)), \quad t \leq 0
    \end{cases}
$$

\par
\textbf{2-3 Ex.14}\\
Let $A \subset S$ be a subset of a regular surface $S$.
Prove that $A$ is itself a regular surface if and only if
$A$ is open in $S$, that is, $A = U \cap S$, where $U$ is
an open set in $\mathbb{R}^3$.

\par
\textbf{\textit{Solution.}}\\
$"\Rightarrow":$\\
Suppose $A$ is a regular surface.\\
$"\Leftarrow":$\\
Suppose $A$ is open in $S$, then there exists 
$U \subset \mathbb{R}^3$ such that $A = U \cap S$
where $U$ is an open set.\\
For each point $p \in A \subset S$, there exists
a parametrization $\mathcal{X}: O \to W \cap S$
satisfying three conditions since $S$ is a regular
surface. Note that $U$ is open so we can assume that
$W$ is sufficiently small such that $W$ is contained
in $A=W\cap S$. Hence $A$ is also a regular surface.

\par
\textbf{2-3 Ex.16}\\
Let $\mathbb{R}^2=\{(x,y,z)\in \mathbb{R}^3:z=-1\}$
be identified with the complex plane $\mathbb{C}$
by setting $(x,y,-1)=x+iy=\xi \in \mathbb{C}$,
let $P:\mathbb{C}\to \mathbb{C}$ be the complex
polynomial
$$
P(\xi)=a_0 \xi^n + a_1 \xi^{n-1} + ... + a_n
$$
where $a_0 \neq 0, a_i \in \mathbb{C}$.
Denote by $\pi_N$ the stereographic projection of
$S^2=\{(x,y,z)\in \mathbb{R}^3: x^2+y^2+z^2=1\}$
from the north pole $N=(0,0,1)$ onto $\mathbb{R}^2$.
Prove that the map $F:S^2 \to S^2$ given by
$$
    F(p)=\pi_N^{-1} \circ P \circ \pi_N(p), \forall p \in S^2 - \{N\}
$$
$$
    F(N)=N
$$
is differentiable.

\par
\textbf{\textit{Solution.}}\\
For $p \in S^2-\{N\}$, it's easy to verify that $F$ is 
differentiable at $p$ since $\pi_N$ is a diffeomorphism
and $P$ is holomorphic.\\
Consider map $G: \mathbb{C} \to \mathbb{C}$ given by
$$
    G(p)=\pi_S \circ \pi_N^{-1} \circ P \circ \pi_N \circ \pi_S^{-1}(p)
$$
where $\pi_S$ is defined similar to $\pi_N$\\
It suffices to show that $G$ is differentiable at $0$.\\
First observe that 
$$
    \pi_N \circ \pi_S^{-1}(\xi)=\frac{1}{\bar{\xi}}, \quad
    \pi_S \circ \pi_N^{-1}(\eta) = \frac{1}{\bar{\eta}}
$$
Hence
$$
\begin{aligned}
    G(\xi)&=\frac{1}{\overline{P\circ \pi_N \circ \pi_S^{-1}(\xi)}}=\frac{1}{\overline{P(\frac{1}{\bar{\xi}})}}\\
    &=\frac{1}{P(\frac{1}{\xi})} = \frac{1}{a_0\xi^n+a_1\xi^{n-1}+...+a_n}=\frac{\xi^n}{a_0+a_1\xi+...+a_n\xi^n}
\end{aligned}
$$
which is differentiable at $0$.\\
Then, since $\pi_S$ is a diffeomorphism,
$$
    F(p) = \pi_S^{-1} \circ G \circ \pi_S
$$
is differentiable at $N$. \qed

\par
\textbf{2-4 Ex.1}\\
Show that the equation of the tangent plane at $(x_0,y_0,z_0)$
of a regular surface given by $f(x,y,z)=0$,where $0$ is a
regular value of $f$,is
$$
    f_x(x_0,y_0,z_0)(x-x_0)+f_y(x_0,y_0,z_0)(y-y_0)+f_z(x_0,y_0,z_0)(z-z_0)=0
$$

\par
\textbf{\textit{Solution.}}\\
Suppose $w$ is a tangent vector of $S=f^{-1}(0)$ at $p=(x_0,y_0,z_0)$
and $\alpha: (-\epsilon, \epsilon) \to S$ is a differentiable
curve such that $\alpha(0)=p$, $\alpha'(0)=w$. Let
$g=f\circ \alpha$, then $g(t)=f(\alpha(t))=0$ for all $t$.
Hence $g'(0)=(f_x(p),f_y(p),f_z(p)) \cdot w = 0$.
Since $w$ is arbitrary, it follows that the equation of the
tangent plane is
$$
    f_x(p)(x-x_0)+f_y(p)(y-y_0)+f_z(p)(z-z_0)=0
$$
\qed

\par
\textbf{2-4 Ex.2}\\
Determine the tangent planes of $x^2+y^2-z^2=1$ at the points
$(x,y,0)$ and show that they are all parallel to the $z$
axis.

\par
\textbf{\textit{Solution.}}\\
Using the conclusion above we know that the equation of
the tangent plane at $(x_0,y_0,0)$ is
$$
    x_0x+y_0y-1=0
$$
Thus the normal vector of the tangent plane is $(x_0,y_0,0)$,
which is normal to $(0,0,1)$, hence $z$ axis is parallel to
the tangent plane at $(x_0,y_0,0)$ for all $x_0,y_0$. \qed

\par
\textbf{2-4 Ex.13}\\
A critical point of a differentiable function $f: S \to \mathbb{R}$
defined on a regular surface $S$ is a point $p \in S$ such
that $df_p=0$.

\par
\textbf{a. }Let $f:S\to \mathbb{R}$ be given by $f(p)=|p-p_0|$,
$p \in S$, $p_0 \notin S$. Show that $p$ is a critical point
of $f$ if and only if the line joining $p$ and $p_0$ is
normal to $S$ at $p$.\\
\textbf{b. }Let $h:S \to \mathbb{R}$ be given by $h(p)=p \cdot v$,
where $v \in \mathbb{R}^3$ is a unit vector. Show that $p \in S$
is a critical point of $f$ if and only if $v$ is a normal
vector of $S$ at $p$.

\par
\textbf{\textit{Solution.}}\\
\textbf{a. }Suppose $p$ is a critical point, then for each $w \in T_p(S)$
$$
    df_p(w)=(\frac{x-x_0}{|p-p_0|}, \frac{y-y_0}{|p-p_0|}, \frac{z-z_0}{|p-p_0|})(w)=\frac{p-p_0}{|p-p_0|}(w)=0
$$
Thus $p-p_0$ is penpendicular to $T_p(S)$ and also $S$.\\
It's easy to verify inversely. \qed

\textbf{b. }\\
Observe that
$$
    dh_p(w) = <v,w>, w \in T_p(S)
$$
It follows that $dh_p=0$ if and only if $v$ is a normal
vector of $S$ at $p$.

\par
\textbf{2-4 Ex.15}\\
Show that if all normals to a connected surface pass through
a fixed point, the surface is contained in a sphere.

\par
\textbf{\textit{Soluiton.}}\\
Suppose the fixed point is denoted by $p_0$, then for each 
point $p \in S$, $p-p_0$ is normal to $T_p(S)$.\\
Let $f(p)=|p-p_0|^2$, then
$$
    df_p(w)=2(p-p_0)(w)=0, \forall w \in T_p(S)
$$
Then we show that $f(p)=C$, for each $p_1,p_2 \in S$,
we can find a curve $\alpha: I \to S$ such that
$\alpha(t_1)=p_1, \alpha(t_2)=p_2$. Consider
$g = f \circ \alpha$, then
$$
    g(t_2)-g(t_1) = \int_{t_1}^{t_2}g'(t)dt
$$
Since $g'(t)=df_{\alpha(t)}(\alpha'(t)) = 0$ for all $t \in I$.
Therefore $g(t_1)=g(t_2)$, i.e. $f(p_1)=f(p_2)$.
Hence $f(p)=C$ for some constant $C$, which implies that
$S \subset \{p\in \mathbb{R}^3:|p-p_0|^2=C\}$. \qed

\par
\textbf{2-4 Ex.16}\\
Let $w$ be a tangent vector to a regular surface $S$ at a
point $p \in S$ and let $\mathcal{X}(u,v)$ and $\bar{\mathcal{X}}(\bar{u}, \bar{v})$
be two parametrizations at $p$. Suppose that the
expressions of $w$ in the bases associated to $\mathcal{X}(u,v)$
and $\bar{\mathcal{X}}(\bar{u}, \bar{v})$ are
$$
    w = \alpha_1 \mathcal{X}_u + \alpha_2 \mathcal{X}_v
$$
and
$$
    w = \beta_1 \bar{\mathcal{X}}_{\bar{u}} +\beta_2 \bar{\mathcal{X}}_{\bar{v}}
$$
Show that the coordinates of $w$ are related by
$$
    \beta_1 = \alpha_1 \frac{\partial{\bar{u}}}{\partial{u}} + \alpha_2 \frac{\partial{\bar{u}}}{\partial{v}}
$$
$$
    \beta_2 = \alpha_1 \frac{\partial{\bar{v}}}{\partial{u}} + \alpha_2 \frac{\partial{\bar{v}}}{\partial{v}}
$$
where $\bar{u}=\bar{u}(u,v)$ and $\bar{v}=\bar{v}(u,v)$
are the expressions of the change of coordinates.

\par
\textbf{\textit{Solution.}}\\
Note that
$$
    (\mathcal{X}_u, \mathcal{X}_v)
    =(\mathcal{\bar{X}}_{\bar{u}},\mathcal{\bar{X}}_{\bar{v}})
    \cdot
    \left(
        \begin{array}{cc}
            \dfrac{\partial{\bar{u}}}{\partial{u}} &\dfrac{\partial{\bar{u}}}{\partial{v}}\\\\
            \dfrac{\partial{\bar{v}}}{\partial{u}} &\dfrac{\partial{\bar{v}}}{\partial{v}}
        \end{array}
    \right)
$$
Hence
$$
(\mathcal{\bar{X}}_{\bar{u}}, \mathcal{\bar{X}}_{\bar{v}}) \cdot 
\left(
    \begin{array}{c}
        \beta_1\\\\
        \beta_2
    \end{array}
\right)
=
(\mathcal{X}_u, \mathcal{X}_v) \cdot 
\left(
    \begin{array}{c}
        \alpha_1\\\\
        \alpha_2
    \end{array}
\right)
=
(\mathcal{\bar{X}}_{\bar{u}},\mathcal{\bar{X}}_{\bar{v}})
    \cdot
    \left(
        \begin{array}{cc}
            \dfrac{\partial{\bar{u}}}{\partial{u}} &\dfrac{\partial{\bar{u}}}{\partial{v}}\\\\
            \dfrac{\partial{\bar{v}}}{\partial{u}} &\dfrac{\partial{\bar{v}}}{\partial{v}}
        \end{array}
    \right)
    \cdot
    \left(
    \begin{array}{c}
        \alpha_1\\\\
        \alpha_2
    \end{array}
\right)
$$
Therefore,
$$
\left(
    \begin{array}{c}
        \beta_1\\\\
        \beta_2
    \end{array}
\right)=
\left(
        \begin{array}{cc}
            \dfrac{\partial{\bar{u}}}{\partial{u}} &\dfrac{\partial{\bar{u}}}{\partial{v}}\\\\
            \dfrac{\partial{\bar{v}}}{\partial{u}} &\dfrac{\partial{\bar{v}}}{\partial{v}}
        \end{array}
    \right)
    \cdot
    \left(
    \begin{array}{c}
        \alpha_1\\\\
        \alpha_2
    \end{array}
\right)
$$

\par
\textbf{2-4 Ex.18}\\
Prove that if a regular surface $S$ meets a plane $P$ in a
single point $p$, then this plane coincides with the tangent
plane of $S$ at $p$.

\par
\textbf{\textit{Solution.}}\\
Suppose the normal vector of $P$ is $n=(a,b,c)\neq 0$.
Then let $f(q)=(q-p)\cdot n$, where $q \in S$.

Assume that $df_p \neq 0$, then there exists some $w \in T_p(S)$
such that $df_p(w) \neq 0$, then we can find $\beta: (-\epsilon, \epsilon) to S$
such that $\beta(0)=p$, $\beta'(0)=w$, let $h = f \circ \beta$,
then $h'(0) = df_p(w) \neq 0$, thus by inverse function theorem,
there exists $t_1, t_2 \in (-\epsilon, \epsilon)$ such that
$h(t_1)h(t_2)<0$. Hence there exists some $t_0$ such that
$h(t_0)=0$. Since $h$ is arbitrary, there are more than one
point in $P \cap S$, leading a contradiction.

Hence $df(p)=0$.
Now for each $w \in T_p(S)$, we can find a curve
$\alpha: (-\epsilon, \epsilon) \to S$ such that 
$\alpha(0)=p, \alpha'(0)=w$.
Now let $g = f \circ \alpha$, then 
$g: (-\epsilon, \epsilon) \to S$ is a differentiable
function and
$$
    g'(0) = \frac{d}{dt}f(\alpha(t))|_{t=0}
    = n \cdot \alpha'(0) = n \cdot w = 0
$$
Therefore $n$ is perpendicular to $Tp(S)$, which implies
that $P$ is $T_p(S)$ exactly.\qed

\par
\textbf{2-4 Ex.19}\\
Let $S \subset \mathbb{R}^3$ be a regular surface and
$P \subset \mathbb{R}^3$ be a plane. If all points of $S$
are on the same side of $P$, prove that $P$ is tangent to 
$S$ at all points of $P \cap S$.

\par
\textbf{\textit{Solution.}}\\
Similarly, given $p \in S \cap P$, define
$$
    f(q) = (q-p) \cdot n
$$
where $n$ is the normal vector of $P$.
Since we know that $S$ is on one side of $P$, without loss
of generality we can assume that $f(q) \geq 0$ for all $q \in S$.\\
For each $p_0 \in S \cap P$, we have 
$f(p_0)=(p_0-p) \cdot n = 0$. It can derive that
$df_{p_0}=0$, otherwise
by inverse function theorem we could find some $q$ such that
$f(q) < 0$.\\
Now pick $w \in T_{p_0}(S)$, we can find a curve $\alpha: (-\epsilon, \epsilon) \to S$
such that $\alpha(0) = p_0$, $\alpha'(0) = w$.\\
Let $g = f \circ \alpha$, then 
$$
g'(0) = df_{p_0}(w) = n \cdot w = 0
$$
Since $p_0$ is arbitrary, $n$ is the normal vector of tangent planes at all
points of $P \cap S$. \qed

\par
\textbf{2-4 Ex.24}\\
(Chain Rule.)Show that if $\phi: S_1 \to S_2$ and 
$\psi: S_2 \to S_3$ are differentiable maps and $p \in S_1$,
then
$$
    d(\psi \circ \phi)_p = d \psi_{\phi(p)} \circ d \phi_p
$$

\par
\textbf{\textit{Solution.}}\\
For each $w \in T_p(S_1)$, we can find a curve
$\alpha: (-\epsilon, \epsilon) \to S_1$ such that
$\alpha(0)=p$, $\alpha'(0)=w$, let $\beta = \phi \circ \alpha$,
$\gamma = \psi \circ \beta$.
By definition of differential,
$$
    \gamma'(0) = d(\psi \circ \phi)_p(w) = d\psi_{\beta(0)}(\beta'(0))
$$
$$
    \beta'(0) = d\phi_p(w), \beta(0)=\phi(\alpha(0))=\phi(p)
$$
Hence
$$
d(\psi \circ \phi)_p(w) = d\psi_{\phi(p)}(d\phi_p(w))
$$
i.e.
$$
d(\psi \circ \phi)_p = d\psi_{\phi(p)} \circ d\phi_p
$$
\qed

\par
\textbf{2-5 Ex.1(a)}\\
Compute the first fundamental form of the following
regular surface:
$$
\mathcal{X}(u,v)=(a \sin u \cos v, b \sin u \sin v, c \cos u)
$$

\par
\textbf{\textit{Solution.}}\\
$$
    \mathcal{X}_u = (a \cos u \cos v, b \cos u \sin v, -c \sin u)
$$
$$
    \mathcal{X}_v = (-a \sin u \sin v, b \sin u \cos v, c \cos u)
$$
E

\end{document}