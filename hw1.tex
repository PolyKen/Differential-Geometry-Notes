\documentclass{article}

\usepackage{changepage,aligned-overset,amsfonts,amssymb,amsthm,enumerate,geometry}
\usepackage{xeCJK}
\geometry{a4paper, scale = 0.8}
\setCJKmainfont{STSong}
\begin{document}
\author{谢铮 15338200}
\title{Homework 1}
\date{Mar 1, 2019}
\maketitle

\setlength\parindent{0em}   % cancel all indent
\setlength\parskip{1.0\baselineskip} % set skip between paragraphs

\par
\textbf{1-2 Ex.2}\\
Let $\alpha(t)$ be a parametrized curve which does not pass through the origin. If $\alpha(t_0)$
is a point of the trace of $\alpha$ closest to the origin and $\alpha'(t_0) \neq 0$, show that the 
position vector $\alpha(t_0)$ is orthogonal to $\alpha'(t_0)$.

\par
\textbf{\textit{Solution.}}\\
Let $s(t) = |\alpha(t)|$, $t_0$ is the minimum point of $s(t)$ since $\alpha(t_0)$ is the closest 
point to the origin on the trace of $\alpha$.
We know that $\alpha(t) = (x(t), y(t), z(t))$ is differentiable and doesn't pass through the origin, so
$$
s(t) = |\alpha(t)| = \sqrt{x^2(t) + y^2(t) + z^2(t)} > 0
$$ 
is also differentiable. Then we have
$$
\begin{aligned}
s'(t) &= \frac{d}{dt}\sqrt{x^2(t) + y^2(t) + z^2(t)}\\ 
&=\frac{x(t)x'(t) + y(t)y'(t) + z(t)z'(t)}{\sqrt{x^2(t) + y^2(t) + z^2(t)}}\\
&=\frac{\alpha(t) \cdot \alpha'(t)}{s(t)}
\end{aligned}
$$
Noticed that $t_0$ is the minimum point of $s(t)$, It follows
$$
s'(t_0) = \frac{\alpha(t_0) \cdot \alpha'(t_0)}{s(t_0)} = 0
$$
which implies $\alpha(t_0) \cdot \alpha'(t_0) = 0$, i.e. $\alpha(t_0)$ is orthogonal to $\alpha'(t_0)$. \quad $\qedsymbol$

\par
\textbf{1-2 Ex.4}\\
Let $\alpha(t): I \to \mathbb{R}^3$ be a parametrized curve and let $v \in \mathbb{R}^3$ be a fixed
vector. Assumed that $\alpha'(t)$ is orthogonal to $v$ for all $t \in I$ and that $\alpha(0)$ is also
orthogonal to $v$. Prove that $\alpha(t)$ is also orthogonal to $v$ for all $t \in I$.

\par
\textbf{\textit{Solution.}}\\
Suppose $I = (a, b), 0 \in (a, b)$, then given $t \in (a,b)$, $\alpha(t)$ can be written as
$$
    \alpha(t) = \int_a^t\alpha'(s)ds
$$
Thus we have
$$
    (\alpha(t) - \alpha(0)) \cdot v = \int_0^t\alpha'(s)ds \cdot v = \int_0^t\alpha'(s) \cdot v ds
$$
It follows
$$
    \quad \alpha(t) \cdot v = \alpha(0) \cdot v + \int_0^t\alpha'(s) \cdot v ds = 0 + \int_0^t0ds = 0 \quad \qedsymbol
$$

\par
\textbf{1-3 Ex.4}\\
Let $\alpha: (0, \pi) \to \mathbb{R}^2$ be given by
$$
    \alpha(t) = (\sin t, \cos t + \log{\tan{\frac{t}{2}}})
$$
where $t$ is the angle that the y axis makes with the vector $\alpha'(t)$. The trace of $\alpha$ is 
called the tractrix. Show that\\
\textbf{a.} $\alpha$ is a differentiable parametrized curve, regular except at $t=\frac{\pi}{2}$.\\
\textbf{b.} The length of the segment of the tangent of the tractrix between the point of tangency and the y
axis is constantly equal to 1.

\par
\textbf{\textit{Solution.}}\\
\textbf{a. }Since $x(t) = \sin t$ and $y(t) = \cos t + \log{\tan{\frac{t}{2}}}$ are both differentiable in $(0, \pi)$,
$\alpha(t)$ is a differentiable map from $(0, \pi)$ to $\mathbb{R}^2$, so $\alpha$ is a differentiable
parametrized curve. Note that
$$
    \alpha'(t) = (\cos t, -\sin t + \frac{1}{\sin t})
$$
Let $|\alpha'(t_0)| = 0$, it follows $\cos t_0 = 0$, $\sin t_0 = \frac{1}{\sin t_0}$ and we have $t_0 = \frac{\pi}{2} + k\pi$, $k \in \mathbb{Z}$.\\
So $t_0 = \frac{\pi}{2}$ is the unique solution in $(0, \pi)$. Therefore, $\alpha$ is regular in $(0, \pi)$ except at $t = \frac{\pi}{2}$. \quad \qedsymbol\\\\
\textbf{b. } Let $(x(t), y(t))$ denote the point of tangency. Since we know that $t$ is the angle that the y axis makes with the vector
$\alpha'(t)$, the segment length $l(t)$ can be calculated by
$$
    \quad l(t) = \frac{x(t)}{\sin t} = \frac{\sin t}{\sin t} = 1 \quad \qedsymbol
$$

\par
\textbf{1-3 Ex.10}\\
(Straight Lines as Shortest) Let $\alpha: I \to \mathbb{R}^3$ be a parametrized curve. Let $[a, b] \subset I$ and set $\alpha(a) = p$, $\alpha(b) = q$.\\
\textbf{a. }Show that, for any constant vector v, $|v| = 1$,
$$
    (q - p) \cdot v = \int_a^b \alpha'(t) \cdot v dt \leq \int_a^b |\alpha'(t)|dt
$$
\textbf{b. }Set
$$
    v = \frac{q - p}{|q - p|}
$$
and show that
$$
    |\alpha(b) - \alpha(a)| \leq \int_a^b |\alpha'(t)|dt
$$
That is, the curve of shortest length from $\alpha(a)$ to $\alpha(b)$ is the straight line joining these points.

\par
\textbf{\textit{Solution.}}\\
\textbf{a. } Since $\alpha$ is differentiable,
$$
    q - p = \alpha(b) - \alpha(a) = \int_a^b \alpha'(t)dt
$$
Thus,
$$
    (q - p) \cdot v = \int_a^b \alpha'(t) dt \cdot v = \int_a^b \alpha'(t) \cdot vdt
$$
For each $t \in (a, b)$, $\alpha'(t) \cdot v \leq |\alpha'(t)||v| = |\alpha'(t)|$, so
$$
    \quad \int_a^b \alpha'(t) \cdot vdt \leq \int_a^b |\alpha'(t)|dt \quad \qedsymbol
$$
\textbf{b. } According to the conclusion above, take $v = \frac{q - p}{|q - p|}$ and
it follows immediately that
$$
    \quad |\alpha(b) - \alpha(a)| = |q - p| = (q - p) \cdot v \leq \int_a^b |\alpha'(t)|dt \quad \qedsymbol
$$

\par
\textbf{1-4 Ex.2}\\
A plane $P$ contained in $\mathbb{R}^3$ is given by the equation $ax+by+cz+d=0$. Show that the vector $v=(a,b,c)$
is perpendicular to the plane and that $\frac{|d|}{\sqrt{a^2+b^2+c^2}}$ measures the distance from the plane to the origin
(0, 0, 0).

\par
\textbf{\textit{Solution.}}\\
For each point $(x, y, z)$ in plane $P$, the equation $ax+by+cz+d=0$ holds. Hence for each vector $u$ contained in $P$,
it can be denoted by $u = (x_2-x_1, y_2-y_1, z_2-z_1)$ where $(x_1, y_1, z_1)$ and $(x_2, y_2, z_2)$ are points in $P$.
Therefore,
$$
    a(x_2-x_1) + b(y_2-y_1) + c(z_2-z_1) = 0
$$
That is,
$$
    u \cdot v = (x_2-x_1, y_2-y_1, z_2-z_1) \cdot (a,b,c) = 0
$$
Suppose $v_0$ is the shortest vector from the origin to $P$, it's easy to see that $v_0$ and $v$ are linear dependent, so
$v_0$ can be written as $\lambda v$, where $\lambda \in \mathbb{R}$, therefore, for each point $(x,y,z) \in P$, 
$$
    ((x,y,z) - v_0) \cdot v_0 = (x-\lambda a, y-\lambda b, z-\lambda c) \cdot \lambda (a,b,c) = 0
$$
i.e.
$$
    \lambda a(x-\lambda a) + \lambda b(y-\lambda b) + \lambda c(x-\lambda c) = -(a^2+b^2+c^2)\lambda^2 + (ax+by+cz)\lambda = 0
$$
$$
    (a^2+b^2+c^2)\lambda^2 + d\lambda = 0
$$
this implies $\lambda=-\frac{d}{a^2+b^2+c^2}$(when $\lambda = 0$, $d = 0$), so $|v_0| = |\lambda||v| = \frac{|d|}{\sqrt{a^2+b^2+c^2}}$,
which is exactly the distance from the plane to the origin (0, 0, 0).

\par
\textbf{1-4 Ex.11}\\
\textbf{a. }Show that the volume $V$ of a parallelepiped generated by three linearly independent vectors $u,v,w \in \mathbb{R}^3$ is
given by $V = |(u \wedge v) \cdot w|$, and introduce an oriented volume in $\mathbb{R}^3$.\\
\textbf{b. }Prove that
$$
    V^2 = 
    \left|\begin{array}{ccc} 
        u \cdot u & u \cdot v & u \cdot w \\ 
        v \cdot u & v \cdot v & v \cdot w \\ 
        w \cdot u & w \cdot v & w \cdot w \\ 
    \end{array}\right|
$$.

\par
\textbf{\textit{Solution.}}\\
\textbf{a. }Let $S$ and $h$ denote the basal area and height of the parallelepiped, then
$$
    V = S \cdot h = |u||v|\sin \langle u,v \rangle h = |u \wedge v|\frac{|(u \wedge v) \cdot w|}{|u \wedge v|} = |(u \wedge v) \cdot w|
    \quad \qedsymbol
$$\\
\textbf{b. }Let
$$
    G = 
    \left[\begin{array}{ccc} 
        u \cdot u & u \cdot v & u \cdot w \\ 
        v \cdot u & v \cdot v & v \cdot w \\ 
        w \cdot u & w \cdot v & w \cdot w \\ 
    \end{array}\right]
$$
If any two vectors of $u, v, w$ are linearly dependent, then it's easy to see both sides will equal to 0. Thus it only remains to verify the cases that $\{u, v, w\}$
are linearly independent.\\
In these cases, $\{u,v,w\}$ is a basis of $\mathbb{R}^3$. By Gram-Schmidt process, we can find an orthonormal basis
$\{\epsilon_i\}$ based on $u,v,w$, and there exists an upper triangular matrix $P$ such that
$$
    (u,v,w) = (\epsilon_1, \epsilon_2, \epsilon_3) P = (\epsilon_1, \epsilon_2, \epsilon_3)
    \left[\begin{array}{ccc} 
        p_{11} & p_{12} & p_{13} \\ 
         & p_{22} & p_{23} \\ 
         &  & p_{33} \\ 
    \end{array}\right]
$$
Hence,
$$
\begin{aligned}
    G &= 
    \left[\begin{array}{ccc} 
        u \cdot u & u \cdot v & u \cdot w \\ 
        v \cdot u & v \cdot v & v \cdot w \\ 
        w \cdot u & w \cdot v & w \cdot w \\ 
    \end{array}\right]
    = 
    \left[\begin{array}{c} 
        u\\ 
        v\\ 
        w\\ 
    \end{array}\right] \cdot
    \left[\begin{array}{ccc}
        u & v & w
    \end{array}\right]\\
    &= P^T \cdot 
    \left[\begin{array}{c} 
        \epsilon_1\\ 
        \epsilon_2\\ 
        \epsilon_3\\ 
    \end{array}\right] \cdot
    \left[\begin{array}{ccc}
        \epsilon_1 & \epsilon_2 & \epsilon_3
    \end{array}\right] \cdot P \\
    &= P^T \cdot I \cdot P = P^TP
\end{aligned}
$$
Therefore, $|G| = |P|^2 = p_{11}^2 \cdot p_{22}^2 \cdot p_{33}^2$.\\
On the other hand,
$$
\begin{aligned}
    V(u,v,w) &= |(u \wedge v) \cdot w|\\
    &= |(p_{11}\epsilon_1 \wedge (p_{12}\epsilon_1 + p_{22}\epsilon_2)) \cdot (p_{13}\epsilon_1 + p_{23}\epsilon_2 + p_{33}\epsilon_3)|\\
    &= |p_{11}p_{22}(\epsilon_1 \wedge \epsilon_2) \cdot (p_{13}\epsilon_1 + p_{23}\epsilon_2 + p_{33}\epsilon_3)|\\
    &= |p_{11}p_{22}\epsilon_3 \cdot (p_{13}\epsilon_1 + p_{23}\epsilon_2 + p_{33}\epsilon_3)|\\
    &= |p_{11}p_{22}p_{33}|
\end{aligned}
$$
So we have $V^2 = p_{11}^2 \cdot p_{22}^2 \cdot p_{33}^2 = |G| = 
\left|\begin{array}{ccc} 
    u \cdot u & u \cdot v & u \cdot w \\ 
    v \cdot u & v \cdot v & v \cdot w \\ 
    w \cdot u & w \cdot v & w \cdot w \\ 
\end{array}\right|$.\quad \qedsymbol

\par
\textbf{1-5 Ex.1}\\
Given the parametrized curve (helix)
$$
    \alpha(s) = (a \cos \frac{s}{c}, a \sin \frac{s}{c}, b \frac{s}{c}), s \in \mathbb{R}
$$
where $c^2 = a^2 + b^2$,\\
\textbf{a. } Show that the parameter $s$ is the arc length.\\
\textbf{b. } Determine the curvature and the tortion of $\alpha$.\\
\textbf{c. } Determine the osculating plane of $\alpha$.\\
\textbf{d. } Show that the lines containing $n(s)$ and passing through $\alpha(s)$
meet the $z$ axis under a constant angle equal to $\frac{\pi}{2}$.\\
\textbf{e. } Show that the tangent line of $\alpha$ make a constant angle with the $z$ axis.\\\\
\textbf{\textit{Solution.}}\\
\textbf{a. } We only need to verify that $|\alpha(s)| \equiv 1$.\\
$$
    \alpha'(s) = (-\frac{a}{c} \sin \frac{s}{c}, \frac{a}{c} \cos \frac{s}{c}, \frac{b}{c})
$$
$$
    \quad |\alpha'(s)| = \sqrt{(-\frac{a}{c} \sin \frac{s}{c})^2 + (\frac{a}{c} \cos \frac{s}{c})^2 + (\frac{b}{c})^2} = \sqrt{\frac{a^2+b^2}{c^2}} = 1 \quad \qedsymbol
$$
\\
\textbf{b. }
$$
    \alpha''(s) = (-\frac{a}{c^2} \cos \frac{s}{c}, -\frac{a}{c^2} \sin \frac{s}{c}, 0)
$$
$$
    k(s) = |\alpha''(s)| = \frac{|a|}{c^2}
$$
$$
    n(s) = (-sgn(a) \cdot \cos \frac{s}{c}, -sgn(a) \cdot \sin \frac{s}{c}, 0)
$$
$$
    b(s) = \alpha'(s) \wedge n(s) = (sgn(a) \cdot \frac{b}{c}\sin \frac{s}{c}, -sgn(a) \cdot \frac{b}{c} \cos \frac{s}{c}, \frac{a}{c})
$$
$$
    b'(s) = (sgn(a) \cdot \frac{b}{c^2}\cos \frac{s}{c}, sgn(a) \cdot \frac{b}{c^2} \sin \frac{s}{c}, 0)
$$
Hence we have $\tau(s) = \frac{b}{c^2}$. \quad \qedsymbol\\\\
\textbf{c. }The osculating plane of $\alpha$ is the plane spanned by $t(s)$ and $n(s)$.
So the normal vector of the osculating plane is $b(s)$. Given $s \in \mathbb{R}$, the osculating plane at $s$ is 
defined by the equation
$$
    sgn(a) \cdot \frac{b}{c}\sin \frac{s}{c}(x - a \cos \frac{s}{c}) - sgn(a) \cdot \frac{b}{c} \cos \frac{s}{c}(y - a \sin \frac{s}{c}) + \frac{a}{c}(z - b\frac{s}{c}) = 0
$$
\qedsymbol
\\\\
\textbf{d. }Note that $n(s) \cdot (0, 0, 1) = 0$. \quad \qedsymbol
\\\\
\textbf{e. }Note that $t(s) \cdot (0, 0, 1) = \frac{b}{c}$ for all $s \in \mathbb{R}$. \quad \qedsymbol

\par
\textbf{1-5 Ex.2}\\
Show that the tortion $\tau$ of $\alpha$ is given by
$$
    \tau(s) = - \frac{\alpha'(s) \wedge \alpha''(s) \cdot \alpha'''(s)}{|k(s)|^2}
$$
\\
\textbf{\textit{Solution.}}\\
Since $b'(s) = \tau(s) n(s)$,
$$
\begin{aligned}
    \tau(s) &= b'(s) \cdot n(s) = (t(s) \wedge n(s))' \cdot n(s) \\
    &= (t'(s) \wedge n(s) + t(s) \wedge n'(s)) \cdot n(s) \\
    &= (t(s) \wedge n'(s)) \cdot n(s) \\
    &= (t(s) \wedge (\frac{\alpha''(s)}{k(s)})') \cdot \frac{\alpha''(s)}{k(s)}\\
    &= \frac{(\alpha'(s) \wedge \alpha'''(s)) \cdot \alpha''(s)}{|k(s)|^2} \\
    &= -\frac{\alpha'(s) \wedge \alpha''(s) \cdot \alpha'''(s)}{|k(s)|^2} \quad \qedsymbol
\end{aligned}
$$
\par
\textbf{1-5 Ex.4}\\
Assume that all normals of a parametrized curve pass through a fixed point. Prove that
the trace of the curve is contained in a circle.\\\\
\textbf{\textit{Solution.}}\\
Suppose the fixed point is denoted by $p_0$. Then given any $s \in I$, $p_0 - \alpha(s) = \lambda(s) \cdot n(s)$, where $0 \leq \lambda(s) \leq 1$.\\
Take derivatives of both sides of the equation, we have
$$
    -t(s) = \lambda(s) n'(s) = \lambda(s) (-k(s)t(s) - \tau(s)b(s)) + \lambda'(s)n(s)
$$
Since $t(s)$ is always perpendicular to $b(s)$ and $n(s)$, it follows that $\tau(s)=0$ and $\lambda'(s)=0$, so $t(s) = \lambda(s) k(s)t(s)$, $\lambda(s) = \frac{1}{k(s)} = r$
where $r$ is a constant.
Therefore, 
$$
    |p_0 - \alpha(s)| = |\lambda(s)n(s)| = r
$$
That is, the trace of $\alpha(s)$ is contained in a circle centered at the point $p_0$ with radius $r$.

\par
\textbf{1-5 Ex.9}\\
Given a differentiable function $k(s)$, $s \in I$, show that the parametrized plane curve having $k(s) = k$
as curvature is given by
$$
    \alpha(s) = (\int \cos \theta(s)ds + a, \int \sin \theta(s)ds + b),
$$
where
$$
    \theta(s) = \int k(s)ds + \phi
$$
\\
\textbf{\textit{Solution.}}\\
Note that
$$
    \alpha''(s) = (-\sin{(\theta(s))} \theta'(s), \cos{(\theta(s))}\theta'(s))
$$
$$
    \quad |\alpha''(s)| = |\theta'(s)| = |k(s)| \quad \qedsymbol
$$

\par
\textbf{1-5 Ex.12}\\
Let $\alpha: I \to \mathbb{R}^3$ be a regular parametrized curve (not necessarily by arc length) and let $\beta: J \to \mathbb{R}^3$ be
a reparametrization of $\alpha(I)$ by the arc length $s = s(t)$, measured from $t_0 \in I$. Let $t = t(s)$
be the inverse function of $s$ and set $\frac{d\alpha}{dt}=\alpha'$, $\frac{d^2\alpha}{dt^2}=\alpha''$, etc.
Prove that\\
\textbf{a. }
$$
    \frac{dt}{ds}=\frac{1}{|\alpha'|}, \quad \frac{d^2t}{ds^2} = -\frac{\alpha' \cdot \alpha''}{|\alpha'|^4}
$$
\textbf{b. }The curvature of $\alpha$ at $t \in I$ is 
$$
    k(t) = \frac{|\alpha' \wedge \alpha''|}{|\alpha'|^3}.\\
$$
\textbf{c. }The tortion of $\alpha$ at $t \in I$ is
$$
    \tau(t) = - \frac{(\alpha' \wedge \alpha'') \cdot \alpha'''}{|\alpha' \wedge \alpha''|^2}.
$$
\textbf{d. }If $\alpha: I \to \mathbb{R}^2$ is a plane curve $\alpha(t) = (x(t), y(t))$, the signed curvature
of $\alpha$ at $t$ is
$$
    k(t) = \frac{x'y'' - x''y'}{(x'^2 + y'^2)^{\frac{3}{2}}}.
$$
\\
\textbf{\textit{Solution.}}\\
\textbf{a. }
$$
    \frac{dt}{ds} = (\frac{ds}{dt})^{-1} = \frac{1}{|\alpha'|}
$$
$$
    \frac{d^2t}{ds^2} = \frac{d}{ds} \frac{1}{|\alpha'(t)|} = -\frac{1}{|\alpha'(t)|^2} \frac{d}{ds}|\alpha'(t)| 
    = -\frac{1}{|\alpha'(t)|^2} \cdot \frac{\alpha'(t) \cdot \alpha''(t)}{|\alpha'(t)|} \frac{dt}{ds} = - \frac{\alpha' \cdot \alpha''}{|\alpha'|^4} \quad \qedsymbol
$$
\\
\textbf{b. } Note that
$$
    \alpha'(t) = \frac{d\beta(s)}{dt} = \frac{d\beta(s)}{ds}\cdot \frac{ds}{dt} = \beta'(s)\frac{ds}{dt}
$$
$$
    \alpha''(t) = \frac{d\alpha'(t)}{dt} = \beta'(s)\frac{d^2s}{dt^2} + \beta''(s)(\frac{ds}{dt})^2
$$
Thus
$$
\begin{aligned}
    |\alpha'(t) \wedge \alpha''(t)| &= |(\beta'(s)\frac{ds}{dt} \wedge (\beta''(s)\frac{ds}{dt})^2)| \\
    &= (\frac{ds}{dt})^3 \cdot |\beta'(s) \wedge \beta''(s)|\\
    &= (|\alpha'(t)|)^3 \cdot k_{\beta}(s(t)) \\
    &= (|\alpha'(t)|)^3 \cdot k_{\alpha}(t)
\end{aligned}
$$
So we have
$$
    k(t) = k_{\alpha}(t) = \frac{|\alpha' \wedge \alpha''|}{|\alpha'|^3} \quad \qedsymbol
$$
\\
\textbf{c. }
$$
\begin{aligned}
    \tau_{\alpha}(t) = \tau_{\beta}(s(t)) &=-\frac{\beta'(s) \wedge \beta''(s) \cdot \beta'''(s)}{|k_{\beta}(s)|^2}\\
    &= -\frac{|\beta'(s)|^6 \cdot (\beta'(s) \wedge \beta''(s)) \cdot \beta'''(s)}{|\beta'(s) \wedge \beta''(s)|^2}\\
    &= -\frac{(\frac{\alpha'}{|\alpha'|} \wedge \frac{\alpha''}{|\alpha'|^2}) \cdot \frac{\alpha'''}{|\alpha'|^3}}{|\frac{\alpha'}{|\alpha'|} \wedge \frac{\alpha''}{|\alpha''|}|^2}\\
    &= -\frac{(\alpha' \wedge \alpha'') \cdot \alpha'''}{|\alpha' \wedge \alpha''|^2} \quad \qedsymbol
\end{aligned}
$$
\\
\textbf{d. }By the conclusion of \textbf{b}, we have
$$
    |k(t)| = \frac{|\alpha' \wedge \alpha''|}{|\alpha'|^3}
    = \frac{|(x', y') \wedge (x'', y'')|}{(x'^2+y'^2)^{\frac{3}{2}}} = \frac{|x'y''-x''y'|}{(x'^2+y'^2)^{\frac{3}{2}}}
$$
According to the definition of signed curvature, $k(t)>0$ when $det(\alpha', \alpha'')>0$, $k(t)<0$ when
$det(\alpha', \alpha'')<0$. Hence
$$
    \quad k(t) = \frac{x'y''-x''y'}{(x'^2+y'^2)^{\frac{3}{2}}} \quad \qedsymbol
$$

\par
\textbf{1-5 Ex.13}\\
Assume that $\tau(s) \neq 0$ and $k'(s) \neq 0$ for all $s \in I$. Show that a necessary and sufficient
condition for $\alpha(I)$ to lie on a sphere is that
$$
    R^2 + (R')^2T^2 = const.
$$
where $R = \frac{1}{k}$, $T = \frac{1}{\tau}$, and $R'$ is the derivative of $R$ relative to $s$.
\\\\
\textbf{\textit{Solution.}}\\
Without loss of generality, we can assume that the sphere is centered at the origin.\\
"$\Rightarrow$":\\
Suppose $\alpha(I)$ lies on a sphere, then there exists some constant $C$ such that
$$
    |\alpha(s)|^2 = C^2
$$
Take derivatives on both sides of the equation, we have
$$
    \alpha(s) \cdot \alpha'(s) = \alpha(s) \cdot t(s) = 0
$$
$$
    \alpha(s) \cdot \alpha''(s) + |\alpha'(s)|^2 = k(s) \alpha(s) \cdot n(s) + 1 = 0
$$
$$
    \alpha(s) \cdot \alpha'''(s) + 3\alpha'(s) \cdot \alpha''(s) = \alpha(s) \cdot \alpha'''(s) = 0
$$
For each $s \in I$, we can write $\alpha(s)$ in the form of
$$
    \alpha(s) = c_1t(s) + c_2n(s) + c_3b(s)
$$
The first equation above implies that $c_1=0$, the second equation implies that $c_2 = -\frac{1}{k(s)}$.\\
Also note that,
$$
\begin{aligned}
    \alpha'''(s) &= (k(s)n(s))' = k'(s)n(s) + k(s)n'(s) = k'(s)n(s) - k^2(s)t(s) - k(s)\tau(s)b(s)\\
    &= -k^2(s) \cdot t(s) + k'(s) \cdot n(s) - k(s)\tau(s) \cdot b(s)
\end{aligned}
$$
Thus the third equation implies that
$$
    c_2 \cdot k'(s) - k(s)\tau(s)c_3 = -\frac{k'(s)}{k(s)} - k(s)\tau(s)c_3 = 0
$$
It follows
$$
    c_3 = -\frac{k'(s)}{k^2(s)\tau(s)}
$$
Thus we have
$$
    \alpha(s) = -\frac{1}{k(s)}n(s) - \frac{k'(s)}{k^2(s)\tau(s)}b(s) = -Rn + R'Tb
$$
And
$$
    \quad |\alpha(s)|^2 = R^2 + (R'T)^2 = C^2 \quad \qedsymbol
$$
\\
"$\Leftarrow$":\\
Let $\beta(s) = \alpha(s) + Rn - R'Tb$.\\
First take derivatives on $R^2 + (R'T)^2 = C^2$, we get
$$
    RR' + (R'T)(R'T)' = 0
$$
Then, note that
$$
\begin{aligned}
    \beta'(s) &= t(s) + R'n + Rn' - (R'T)'b - (R'T)b' \\
    &= t + R'n + R(-kt-\tau b) - (R'T)'b - (R'T)\tau n\\
    &= t + R'n - R(\frac{t}{R} + \frac{b}{T}) - (R'T)'b - (R'T)\frac{n}{T}\\
    &= t + R'n - t - \frac{R}{T}b - (R'T)'b - R'n\\
    &= -\frac{R}{T}b - (R'T)'b = -b(\frac{R}{T} + (R'T)')
\end{aligned}
$$
Hence
$$
    \beta'(s) \cdot R'T = -b(RR' + (R'T)(R'T)') = 0
$$
Since $k' \neq 0$, $\tau \neq 0$, it implies $\beta'(s) = 0$ and thus $\beta(s)$ is
a constant $p_0 \in \mathbb{R}^3$. So we have
$$
    \quad |\alpha - p_0| = |\alpha - \beta| = C \quad \qedsymbol
$$

\end{document}