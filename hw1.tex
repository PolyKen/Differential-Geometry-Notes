\documentclass{article}

\usepackage{changepage,aligned-overset,amsfonts,amssymb,amsthm,enumerate,geometry}
\usepackage{xeCJK}
\geometry{a4paper, scale = 0.8}
\setCJKmainfont{STSong}
\begin{document}
\author{谢铮 15338200}
\title{Homework 1}
\date{Mar 1, 2019}
\maketitle

\setlength\parindent{0em}   % cancel all indent
\setlength\parskip{1.0\baselineskip} % set skip between paragraphs

\par
\textbf{1-2 Ex.2}\\
Let $\alpha(t)$ be a parametrized curve which does not pass through the origin. If $\alpha(t_0)$
is a point of the trace of $\alpha$ closest to the origin and $\alpha'(t_0) \neq 0$, show that the 
position vector $\alpha(t_0)$ is orthogonal to $\alpha'(t_0)$.

\par
\textbf{\textit{Solution.}}\\
Let $s(t) = |\alpha(t)|$, $t_0$ is the minimum point of $s(t)$ since $\alpha(t_0)$ is the closest 
point to the origin on the trace of $\alpha$.
We know that $\alpha(t) = (x(t), y(t), z(t))$ is differentiable and doesn't pass through the origin, so
$$
s(t) = |\alpha(t)| = \sqrt{x^2(t) + y^2(t) + z^2(t)} > 0
$$ 
is also differentiable. Then we have
$$
\begin{aligned}
s'(t) &= \frac{d}{dt}\sqrt{x^2(t) + y^2(t) + z^2(t)}\\ 
&=\frac{x(t)x'(t) + y(t)y'(t) + z(t)z'(t)}{\sqrt{x^2(t) + y^2(t) + z^2(t)}}\\
&=\frac{\alpha(t) \cdot \alpha'(t)}{s(t)}
\end{aligned}
$$
Noticed that $t_0$ is the minimum point of $s(t)$, It follows
$$
s'(t_0) = \frac{\alpha(t_0) \cdot \alpha'(t_0)}{s(t_0)} = 0
$$
which implies $\alpha(t_0) \cdot \alpha'(t_0) = 0$, i.e. $\alpha(t_0)$ is orthogonal to $\alpha'(t_0)$. \quad $\qedsymbol$

\par
\textbf{1-2 Ex.4}\\
Let $\alpha(t): I \to \mathbb{R}^3$ be a parametrized curve and let $v \in \mathbb{R}^3$ be a fixed
vector. Assumed that $\alpha'(t)$ is orthogonal to $v$ for all $t \in I$ and that $\alpha(0)$ is also
orthogonal to $v$. Prove that $\alpha(t)$ is also orthogonal to $v$ for all $t \in I$.

\par
\textbf{\textit{Solution.}}\\
Suppose $I = (a, b), 0 \in (a, b)$, then $\alpha(t)$ can be written as
$$
    \alpha(t) = \int_a^t\alpha'(s)ds, \quad a < t < b
$$
Thus we have
$$
    (\alpha(t) - \alpha(0)) \cdot v = \int_0^t\alpha'(s)ds \cdot v = \int_0^t\alpha'(s) \cdot v ds
$$
It follows
$$
    \quad \alpha(t) \cdot v = \alpha(0) \cdot v + \int_0^t\alpha'(s) \cdot v ds = 0 + \int_0^t0ds = 0 \quad \qedsymbol
$$

\par
\textbf{1-3 Ex.4}\\
Let $\alpha: (0, \pi) \to \mathbb{R}^2$ be given by
$$
    \alpha(t) = (\sin t, \cos t + \log{\tan{\frac{t}{2}}})
$$
where $t$ is the angle that the y axis makes with the vector $\alpha'(t)$. The trace of $\alpha$ is 
called the tractrix. Show that\\
\textbf{a.} $\alpha$ is a differentiable parametrized curve, regular except at $t=\frac{\pi}{2}$.\\
\textbf{b.} The length of the segment of the tangent of the tractrix between the point of tangency and the y
axis is constantly equal to 1.

\par
\textbf{\textit{Solution.}}\\
\textbf{a. }Since $x(t) = \sin t$ and $y(t) = \cos t + \log{\tan{\frac{t}{2}}}$ are both differentiable in $(0, \pi)$,
$\alpha(t)$ is a differentiable map from $(0, \pi)$ to $\mathbb{R}^2$, so $\alpha$ is a differentiable
parametrized curve. Note that
$$
    \alpha'(t) = (\cos t, -\sin t + \frac{1}{\sin t})
$$
Let $|\alpha'(t_0)| = 0$, it follows $\cos t_0 = 0$, $\sin t_0 = \frac{1}{\sin t_0}$ and we have $t_0 = \frac{\pi}{2} + k\pi$, $k \in \mathbb{Z}$.\\
So $t_0 = \frac{\pi}{2}$ is the only solution in $(0, \pi)$. Therefore, $\alpha$ is regular in $(0, \pi)$ except at $t = \frac{\pi}{2}$. \quad \qedsymbol\\\\
\textbf{b. } Let $(x(t), y(t))$ denote the point of tangency. Since we know that $t$ is the angle that the y axis makes with the vector
$\alpha'(t)$, the segment length can be calculated by
$$
    \quad l(t) = \frac{x(t)}{\sin t} = \frac{\sin t}{\sin t} = 1 \quad \qedsymbol
$$

\par
\textbf{1-3 Ex.10}\\
(Straight Lines as Shortest) Let $\alpha: I \to \mathbb{R}^3$ be a parametrized curve. Let $[a, b] \subset I$ and set $\alpha(a) = p$, $\alpha(b) = q$.\\
\textbf{a. }Show that, for any constant vector v, $|v| = 1$,
$$
    (q - p) \cdot v = \int_a^b \alpha'(t) \cdot v dt \leq \int_a^b |\alpha'(t)|dt
$$
\textbf{b. }Set
$$
    v = \frac{q - p}{|q - p|}
$$
and show that
$$
    |\alpha(b) - \alpha(a)| \leq \int_a^b |\alpha'(t)|dt
$$
That is, the curve of shortest length from $\alpha(a)$ to $\alpha(b)$ is the straight line joining these points.

\par
\textbf{\textit{Solution.}}\\
\textbf{a. } Since $\alpha$ is differentiable,
$$
    q - p = \alpha(b) - \alpha(a) = \int_a^b \alpha'(t)dt
$$
Thus,
$$
    (q - p) \cdot v = \int_a^b \alpha'(t) dt \cdot v = \int_a^b \alpha'(t) \cdot vdt
$$
For each $t \in (a, b)$, $\alpha'(t) \cdot v \leq |\alpha'(t)||v| = |\alpha'(t)|$, so
$$
    \quad \int_a^b \alpha'(t) \cdot vdt \leq \int_a^b |\alpha'(t)|dt \quad \qedsymbol
$$
\textbf{b. } According to the conclusion above, take $v = \frac{q - p}{|q - p|}$ and
it follows immediately that
$$
    \quad |\alpha(b) - \alpha(a)| = |q - p| = (q - p) \cdot v \leq \int_a^b |\alpha'(t)|dt \quad \qedsymbol
$$

\par
\textbf{1-4 Ex.2}\\
A plane $P$ contained in $\mathbb{R}^3$ is given by the equation $ax+by+cz+d=0$. Show that the vector $v=(a,b,c)$
is perpendicular to the plane and that $\frac{|d|}{\sqrt{a^2+b^2+c^2}}$ measures the distance from the plane to the origin
(0, 0, 0).
\textbf{\textit{Proof.}}\\
For each point $(x, y, z)$ in plane $P$, the equation $ax+by+cz+d=0$ holds. Hence for each vector $u$ contained in $P$,
it can be denoted by $u = (x_2-x_1, y_2-y_1, z_2-z_1)$ where $(x_1, y_1, z_1)$ and $(x_2, y_2, z_2)$ are points in $P$.
Therefore,
$$
    a(x_2-x_1) + b(y_2-y_1) + c(z_2-z_1) = 0
$$
That is,
$$
    v \cdot u = (a, b, c) \cdot (x_2-x_1, y_2-y_1, z_2-z_1) = 0
$$
Suppose $v_0$ is the shortest vector from the origin to $P$, it's easy to see that $v_0$ and $v$ are linear dependent, so
$v_0$ can be written as $\lambda v$, where $\lambda \in \mathbb{R}$, therefore, for each point $(x,y,z) \in P$, 
$$
    ((x,y,z) - v_0) \cdot v_0 = (x-\lambda a, y-\lambda b, z-\lambda c) \cdot \lambda (a,b,c) = 0
$$
i.e.
$$
    \lambda a(x-\lambda a) + \lambda b(y-\lambda b) + \lambda c(x-\lambda c) = -(a^2+b^2+c^2)\lambda^2 + (ax+by+cz)\lambda = 0
$$
$$
    (a^2+b^2+c^2)\lambda^2 + d\lambda = 0
$$
this implies $\lambda=-\frac{d}{a^2+b^2+c^2}$(when $\lambda = 0$, $d = 0$), so $|v_0| = |\lambda||v| = \frac{|d|}{\sqrt{a^2+b^2+c^2}}$,
which is exactly the distance from the plane to the origin (0, 0, 0).

\par
\textbf{1-4 Ex.11}\\
\textbf{a. }Show that the volume $V$ of a parallelepiped generated by three linearly independent vectors $u,v,w \in \mathbb{R}^3$ is
given by $V = |(u \wedge v) \cdot w|$, and introduce an oriented volume in $\mathbb{R}^3$.\\
\textbf{b. }Prove that
$$
    V^2 = 
    \left|\begin{array}{ccc} 
        u \cdot u & u \cdot v & u \cdot w \\ 
        v \cdot u & v \cdot v & v \cdot w \\ 
        w \cdot u & w \cdot v & w \cdot w \\ 
    \end{array}\right|
$$.

\par
\textbf{\textit{Proof.}}\\
\textbf{a. }
$$
    V = S \cdot h = |u||v|\sin \langle u,v \rangle h = |u \wedge v| \rangle \frac{|(u \wedge v) \cdot w|}{|u \wedge v|} = |u \wedge v| \cdot w 
    \quad \qedsymbol
$$\\
\textbf{b. }Note that both sides of the equation
$$
    V^2 = 
    \left|\begin{array}{ccc} 
        u \cdot u & u \cdot v & u \cdot w \\ 
        v \cdot u & v \cdot v & v \cdot w \\ 
        w \cdot u & w \cdot v & w \cdot w \\ 
    \end{array}\right|
$$
are linear in $u,v,w$. So it suffices to show that the equation holds for all basis vectors $e_1, e_2, e_3$, moreover, it's easy to see that,
if $u, v, w$ are linearly dependent, then both sides will equal to 0. Thus it only remains to verify the cases that $(u, v, w)$ is a 
substitution of $(e_1, e_2, e_3)$. In these cases,
$$
    V^2 = |(u \wedge v) \cdot w|^2 = 1
$$
$$
    \quad
    \left|\begin{array}{ccc} 
    u \cdot u & u \cdot v & u \cdot w \\ 
    v \cdot u & v \cdot v & v \cdot w \\ 
    w \cdot u & w \cdot v & w \cdot w \\ 
    \end{array}\right|
    =
    \left|\begin{array}{ccc} 
    1 & 0 & 0 \\ 
    0 & 1 & 0 \\ 
    0 & 0 & 1 \\ 
    \end{array}\right|
    = 1 \quad \qedsymbol
$$


\end{document}